\documentclass[12pt]{article}

% --- Paquetes ---
\usepackage{tikz}
\usepackage{pgfplots}
\pgfplotsset{compat=1.18}
\usepackage[most]{tcolorbox}
\tcbuselibrary{ams math}
\usepackage[spanish,es-tabla]{babel}   % español
\usepackage[utf8]{inputenc}            % acentos
\usepackage[T1]{fontenc}
\usepackage{lmodern}
\usepackage{geometry}
\usepackage{fancyhdr}
\usepackage{xcolor}
\usepackage{titlesec}
\usepackage{lastpage}
\usepackage{amsmath,amssymb}
\usepackage{enumitem}
\usepackage[table]{xcolor} % para \cellcolor y \rowcolor
\usepackage{colortbl}      % colores en tablas
\usepackage{float}         % para usar [H] si quieres fijar la tabla
\usepackage{array}         % mejor control de columnas
\usepackage{amssymb}       % para palomita
\usepackage{graphicx}      % para logo github
\usepackage{hyperref}
\usepackage{setspace} % para hipervinculo
\usepackage[normalem]{ulem}
\usepackage{siunitx}       % Asegúrate de tener este paquete en el preámbulo
\usepackage{booktabs}
\sisetup{
    output-decimal-marker = {.},
    group-separator = {,},
    group-minimum-digits = 4,
    detect-all
}

% Etiqueta en el caption (en la tabla misma)
\usepackage{caption}
\captionsetup[table]{name=Tabla, labelfont=bf, labelsep=period}

% Prefijo en la *Lista de tablas*
\usepackage{tocloft}
\renewcommand{\cfttabpresnum}{Tabla~} % texto antes del número
\renewcommand{\cfttabaftersnum}{.}    % punto después del número
\setlength{\cfttabnumwidth}{5em}      % ancho para "Tabla 10." ajusta si hace falta



% --- Márgenes y encabezado ---
\geometry{left=1in, right=1in, top=1in, bottom=1in}

% Alturas del encabezado (un poco más por las 2–3 líneas del header)
\setlength{\headheight}{32pt}
\setlength{\headsep}{20pt}

\definecolor{maroon}{RGB}{128, 0, 0}

\pagestyle{fancy}
\fancyhf{}

% Regla del encabezado (opcional)
\renewcommand{\headrulewidth}{0.4pt}

% Encabezado izquierdo
\fancyhead[L]{%
  \textcolor{maroon}{\textbf{El Colegio de México}}\\
  \textbf{Macroeconomics 3}
}

% Encabezado derecho
\fancyhead[R]{%
  Homework 1\\
  \textbf{Jose Daniel Fuentes García}\\
  Github : \includegraphics[height=1em]{github.png}~\href{https://github.com/danifuentesga}{\texttt{danifuentesga}}
}

% Número de página al centro del pie
\fancyfoot[C]{\thepage}

% --- APLICAR EL MISMO ESTILO A PÁGINAS "PLAIN" (TOC, LOT, LOF) ---
\fancypagestyle{plain}{%
  \fancyhf{}
  \renewcommand{\headrulewidth}{0.4pt}
  \fancyhead[L]{%
    \textcolor{maroon}{\textbf{El Colegio de México}}\\
    \textbf{Macroeconomics 3}
  }
  \fancyhead[R]{%
    Homework 1:The Intertemporal Current Account Model\\
    \textbf{Jose Daniel Fuentes García}\\
    Github : \includegraphics[height=1em]{github.png}~\href{https://github.com/danifuentesga}{\texttt{danifuentesga}}
  }
  \fancyfoot[C]{\thepage}
}

% Pie de página centrado
\fancyfoot[C]{\thepage\ de \pageref{LastPage}}

\renewcommand{\headrulewidth}{0.4pt}

% --- Color principal ---
\definecolor{formalblue}{RGB}{0,51,102} % azul marino sobrio

% --- Estilo de títulos ---
\titleformat{\section}[hang]{\bfseries\Large\color{formalblue}}{}{0em}{}[\titlerule]
\titleformat{\subsection}{\bfseries\large\color{formalblue}}{\thesubsection}{1em}{}


% --- Listas ---
\setlist[itemize]{leftmargin=1.2em}

% --- Sin portada ---
\title{}
\author{}
\date{}

\begin{document}

\begin{titlepage}
    \vspace*{-1cm}
    \noindent
    \begin{minipage}[t]{0.49\textwidth}
        \includegraphics[height=2.2cm]{colmex.jpg}
    \end{minipage}%
    \begin{minipage}[t]{0.49\textwidth}
        \raggedleft
        \includegraphics[height=2.2cm]{cee.jpg}
    \end{minipage}

    \vspace*{2cm}

    \begin{center}
        \Huge \textbf{CENTRO DE ESTUDIOS ECONÓMICOS} \\[1.5em]
        \Large Maestría en Economía 2024--2026 \\[2em]
        \Large Macroeconomics 3 \\[3em]
        \LARGE \textbf{Homework 1: The Intertemporal Current Account Model} \\[3em]
        \large \textbf{PRESENTED BY:} José Daniel Fuentes García \\[3em]
        \large \textbf{PROFESSOR:} Stephen McKnight \\[0.9em]
        
    \end{center}

    \vfill
\end{titlepage}

\newpage

\setcounter{secnumdepth}{2}
\setcounter{tocdepth}{3}
\tableofcontents


\newpage

\section*{\noindent\textbf{Problem 1}}
\addcontentsline{toc}{section}{Problem 1}

\doublespacing

Consider a \textbf{two-period small open endowment economy}. Suppose that the representative agent has \textbf{lifetime utility} given by:

\[
U = \log(C_1) + \beta \log(C_2),
\]

where \( C_t \) is the \textbf{consumption level} in period \( t = \{1,2\} \), and \( 0 < \beta < 1 \) is the \textbf{subjective discount factor}. In each period, the agent receives with certainty an \textbf{endowment} \( Y_t \). Initially assume that there is \textbf{no government consumption} in the economy (\( G_t = 0 \)).


\subsection*{\noindent\textbf{a)}}
\addcontentsline{toc}{subsection}{a)}

If \( r \) denotes the \textbf{exogenously given world real interest rate} for borrowing and lending from the world financial market, derive the \textbf{period 1 and period 2 budget constraints} and the \textbf{intertemporal budget constraint}. Assume that the economy’s \textbf{initial net foreign assets are zero}.

% --- Definir color rojo formal ---
\definecolor{formalred}{RGB}{178,34,34} % rojo tipo "firebrick"

\vspace{0.5em}
\noindent\textcolor{formalred}{\textbf{ANSWER:}}

\singlespacing


\textbf{Assumptions:}
\begin{itemize}
  \item \( C_t \): consumption in period \( t \), \( t = 1, 2 \)  
  \item \( Y_t \): endowment (income) in period \( t \)
  \item \( B_t \): NFA in \( t \) 
  \item \( r \): world interest rate
\end{itemize}

\textbf{Initial condition:} \( B_0 = 0 \) \\
\textbf{Terminal condition:} \( B_2 = 0 \)

\vspace{0.5em}
\textbf{Period 1 Budget Constraint: [ Consumption = Income ]}

\[
\fcolorbox{red}{white}{$
  C_1 + B_1 = Y_1 + (1 + r)B_0
$}
\]

Since \( B_0 = 0 \), this simplifies to:
\[
C_1 + B_1 = Y_1 \tag{1}
\]

\vspace{0.5em}
\textbf{Period 2 Budget Constraint:}
\[
C_2 + B_2 = Y_2 + (1 + r)B_1 \quad
\]
Since \( B_2 = 0 \), this becomes:
\[
C_2 = Y_2 + (1 + r)B_1 \tag{2}
\]

From equation (1), solve for \( B_1 \):
\[
B_1 = Y_1 - C_1 \tag{3}
\]

Substitute (3) into (2):
\[
C_2 = Y_2 + (1 + r)(Y_1 - C_1) \tag{4}
\]

Expanding:
\[
C_2 = Y_2 + (1 + r)Y_1 - (1 + r)C_1 \tag{5}
\]

Rearranging terms:
\[
\fcolorbox{red}{white}{$
(1 + r)C_1 + C_2 = (1 + r)Y_1 + Y_2
$} \tag{6}
\]


Divide both sides by \( (1 + r) \) to get the \textbf{intertemporal budget constraint}:

\[
\fcolorbox{red}{white}{$
\displaystyle
C_1 + \frac{C_2}{1 + r} = Y_1 + \frac{Y_2}{1 + r}
$} \quad \text{[Intertemporal BC]}
\]

\subsection*{\noindent\textbf{b)}}
\addcontentsline{toc}{subsection}{b)}

Given the budget constraints derived above, solve the agent’s maximization problem.

\vspace{0.5em}
\noindent\textcolor{formalred}{\textbf{ANSWER:}}

\textbf{Intertemporal Budget Constraint (from above):}

\[
C_1 + \frac{C_2}{1 + r} = Y_1 + \frac{Y_2}{1 + r} \quad \text{[Intertemporal BC]} \tag{1}
\]


\textbf{Maximization problem:}
\[
\max_{C_1, C_2} \log C_1 + \beta \log C_2 \quad \text{s.t. constraint (1)} \tag{2}
\]

\textbf{Lagrangian:}
\[
\mathcal{L} = \log C_1 + \beta \log C_2 - \lambda \left( C_1 + \frac{C_2}{1 + r} - Y_1 - \frac{Y_2}{1 + r} \right) \tag{3}
\]

\textbf{First-order conditions:}
\begin{align*}
\frac{\partial \mathcal{L}}{\partial C_1} &: \quad \frac{1}{C_1} - \lambda = 0 \Rightarrow \lambda = \frac{1}{C_1} \tag{4} \\
\frac{\partial \mathcal{L}}{\partial C_2} &: \quad \frac{\beta}{C_2} - \lambda \cdot \frac{1}{1 + r} = 0 \Rightarrow \lambda = \frac{\beta(1 + r)}{C_2} \tag{5}
\end{align*}

\textbf{Euler Equation (from (4) = (5)):}
\[
\frac{1}{C_1} = \frac{\beta(1 + r)}{C_2} \Rightarrow C_2 = \beta(1 + r)C_1 \tag{6}
\]

\textbf{Substitute (6) into (1):}
\[
C_1 + \frac{\beta(1 + r)C_1}{1 + r} = Y_1 + \frac{Y_2}{1 + r} \Rightarrow (1 + \beta)C_1 = Y_1 + \frac{Y_2}{1 + r} \tag{7}
\]

\vspace{0.5em}
\textbf{Optimal } \( C_1^* \):

\[
\fcolorbox{red}{white}{$
\displaystyle
C_1^* = \dfrac{(1 + r)Y_1 + Y_2}{(1 + \beta)(1 + r)}
$} \tag{8}
\]


\textbf{Optimal } \( C_2^* \) (using equation (6)):

\[
\fcolorbox{red}{white}{$
\displaystyle
C_2^* = \beta(1 + r)C_1^* = \frac{\beta\left( (1 + r)Y_1 + Y_2 \right)}{1 + \beta}
$} \tag{9}
\]

\textbf{Optimal asset holdings } \( B_1^* = Y_1 - C_1^* \):

\[
\fcolorbox{red}{white}{$
\displaystyle
B_1^* = Y_1 - C_1^* = \frac{\beta Y_1 - \frac{Y_2}{1 + r}}{1 + \beta}
= \frac{\beta(1 + r)Y_1 - Y_2}{(1 + \beta)(1 + r)}
$} \tag{10}
\]


\subsection*{\noindent\textbf{c)}}
\addcontentsline{toc}{subsection}{c)}


If \( \beta = 0.8 \), \( r = 0.25 \), \( Y_1 = 5 \) and \( Y_2 = 10 \), derive the optimal consumption level in each period.

\vspace{0.5em}
\noindent\textcolor{formalred}{\textbf{ANSWER:}}

\textbf{Optimal } \( C_1^* \):
\begin{align*}
C_1^* &= \frac{Y_1 + \dfrac{Y_2}{1 + r}}{1 + \beta} \\
C_1^* &= \frac{5 + \dfrac{10}{1.25}}{1 + 0.8} \\
C_1^* &= \frac{5 + 8}{1.8} \\
C_1^* &= \frac{13}{1.8} \\
&\fcolorbox{red}{white}{$
\displaystyle C_1^* = \frac{65}{9} \approx 7.22
$}
\end{align*}

\vspace{1em}
\textbf{Optimal } \( C_2^* \):
\begin{align*}
C_2^* &= \beta(1 + r)C_1^* \\
C_2^* &= 0.8 \cdot 1.25 \cdot 7.22 \\
C_2^* &= 1 \cdot 7.22 \\
&\fcolorbox{red}{white}{$
\displaystyle C_2^* = \frac{65}{9} \approx 7.22
$}
\end{align*}

Both consumption levels are the same because preferences and the interest rate align so that the agent perfectly \textbf{smooths consumption} across periods.

\subsection*{\noindent\textbf{d)}}
\addcontentsline{toc}{subsection}{d)}

Define the current account for this economy. Using the results obtained from \textbf{(c)}
above, calculate the current account in each period.

\vspace{0.5em}
\noindent\textcolor{formalred}{\textbf{ANSWER:}}

\singlespacing

\textbf{From the period budget constraint:}
\[
C_t + B_t = Y_t + (1 + r)B_{t-1}
\]

\textbf{Rearranged:}
\[
B_t - B_{t-1} = Y_t - C_t + r B_{t-1}
\]

\textbf{Define the current account:}
\[
CA_t \equiv B_t - B_{t-1}
\]

\textbf{Substitute:}
\[
CA_t = Y_t - C_t + r B_{t-1}
\]

\vspace{1em}
\textbf{Using:} \( \beta = 0.8, \quad r = 0.25, \quad Y_1 = 5, \quad Y_2 = 10, \quad C_1 = C_2 = \dfrac{65}{9} \)

\vspace{1em}
\textbf{Solve for } \( B_1 \):
\[
B_1 = Y_1 - C_1 = 5 - \frac{65}{9} = \boxed{-\frac{20}{9}}
\]

\vspace{1em}
\textbf{Period 1 Current Account:}
\begin{align*}
CA_1 &= Y_1 - C_1 + r B_0 \\
CA_1 &= 5 - \frac{65}{9} + 0.25 \cdot 0 \\

\[
\fcolorbox{red}{white}{$
\displaystyle CA_1 = -\frac{20}{9} \approx -2.22
$}
\]
\end{align*}

\vspace{1em}
\textbf{Period 2 Current Account:}
\begin{align*}
CA_2 &= Y_2 - C_2 + r B_1 \\
CA_2 &= 10 - \frac{65}{9} + 0.25 \cdot \left(-\frac{20}{9}\right) \\

\[
\fcolorbox{red}{white}{$
\displaystyle CA_2 = \frac{20}{9} \approx 2.22
$}
\]

\end{align*}

\textbf{Checks:}

\textbf{Current Account (CA):}

\[
CA_1 = Y_1 - C_1 \quad \text{(since } B_0 = 0\text{)}
\]

\[
CA_2 = Y_2 - C_2 + r B_1
\]

\vspace{1em}
\textbf{Trade Balance (TB):}

\[
TB_1 = Y_1 - C_1, \qquad TB_2 = Y_2 - C_2
\]

\vspace{1em}
\textbf{Intertemporal consistency:}

\[
CA_2 = -CA_1
\]

\textit{(Because } \( CA_1 + CA_2 = B_2 - B_0 = 0 \) \textit{ with } \( B_0 = B_2 = 0 \) \textit{)}


\subsection*{\noindent\textbf{e)}}
\addcontentsline{toc}{subsection}{e)}

Define the autarky interest rate and use the parameter values given in \textbf{c)} above to
calculate its value.

\vspace{0.5em}
\noindent\textcolor{formalred}{\textbf{ANSWER:}}

\textbf{Definition (Autarky Interest Rate):}

The autarky interest rate \( r_A \) is such that there is no trade, i.e.:
\[
B_1 = 0 \quad \Rightarrow \quad C_1 = Y_1, \quad C_2 = Y_2
\]

\textbf{Euler Equation:}
\[
C_2 = \beta(1 + r)C_1
\]

\textbf{Autarky condition:}
\[
Y_2 = \beta(1 + r_A)Y_1
\]

\textbf{Solve for } \( r_A \):
\begin{align*}
1 + r_A &= \frac{Y_2}{\beta Y_1} \\
r_A &= \frac{Y_2}{\beta Y_1} - 1
\end{align*}

\textbf{Substitute: } \( \beta = 0.8, \quad Y_1 = 5, \quad Y_2 = 10 \)
\begin{align*}
r_A &= \frac{10}{0.8 \cdot 5} - 1 \\
r_A &= \frac{10}{4} - 1 \\
r_A &= 2.5 - 1 \\
&\fcolorbox{red}{white}{$\displaystyle r_A = 1.5$}
\end{align*}

\vspace{1em}
\textbf{Check:} Use optimal consumption formula with \( r_A \)

\textbf{Check } \( C_1^* \):
\begin{align*}
C_1^* &= \frac{Y_1 + \dfrac{Y_2}{1 + r_A}}{1 + \beta} \\
C_1^* &= \frac{5 + \dfrac{10}{2.5}}{1.8} \\
C_1^* &= \frac{5 + 4}{1.8} \\
C_1^* &= \frac{9}{1.8} \\
C_1^* &= 5 = Y_1
\end{align*}

\textbf{Check } \( C_2^* \):
\begin{align*}
C_2^* &= \beta(1 + r_A)C_1^* \\
C_2^* &= 0.8 \cdot 2.5 \cdot 5 \\
C_2^* &= 10 = Y_2
\end{align*}

\textbf{Check } \( B_1 = Y_1 - C_1^* = 0 \)


\subsection*{\noindent\textbf{f)}}
\addcontentsline{toc}{subsection}{f)}

Depict your results showing equilibrium consumption levels under autarky, equilibrium
consumption levels for the small open economy and the current account for both
periods.

\vspace{0.5em}
\noindent\textcolor{formalred}{\textbf{ANSWER:}}

\begin{figure}[H]
    \centering
    {\captionsetup{font=small}
    \caption{r Autrky vs r Open Economy}  % ← afuera del grupo
    \vspace{0.3em}
    
    \includegraphics[width=0.90\textwidth]{grafi1.jpg}
    
    \caption*{Fuente: Elaboración propia} 

\end{figure}


\subsection*{\noindent\textbf{g)}}
\addcontentsline{toc}{subsection}{g)}

Assume now that there is a temporary increase in government spending. Specifically, suppose that the government consumes the amount \( G_1 = 2 \) in period 1 and nothing in period 2 (\( G_2 = 0 \)).

\vspace{0.5em}
\noindent\textcolor{formalred}{\textbf{ANSWER:}}

\textbf{New budget constraint (with government spending):}
\[
C_1 + \frac{C_2}{1 + r} = (Y_1 - G_1) + \frac{Y_2 - G_2}{1 + r}
\]

\textbf{Given:}
\[
\beta = 0.8, \quad r = 0.25, \quad Y_1 = 5, \quad Y_2 = 10, \quad G_1 = 2, \quad G_2 = 0
\]
\[
Y_1 - G_1 = 3, \quad Y_2 - G_2 = 10
\]

\textbf{Utility:} \( U = \log C_1 + \beta \log C_2 \)

\textbf{Optimal consumption (as in part b, replacing \( Y_t \) with \( Y_t - G_t \)):}
\begin{align*}
C_1^* &= \frac{Y_1 - G_1 + \dfrac{Y_2 - G_2}{1 + r}}{1 + \beta} \\
C_1^* &= \frac{3 + \dfrac{10}{1.25}}{1.8} \\
C_1^* &= \frac{3 + 8}{1.8} \\
C_1^* &= \frac{11}{1.8} = \frac{55}{9} \approx 6.11 \\
&\fcolorbox{red}{white}{$\displaystyle C_1^* = \frac{55}{9} \approx 6.11$}
\end{align*}

\vspace{1em}
\begin{align*}
C_2^* &= \beta(1 + r) C_1^* \\
C_2^* &= 0.8 \cdot 1.25 \cdot \frac{55}{9} \\
C_2^* &= \frac{55}{9} \approx 6.11 \\
&\fcolorbox{red}{white}{$\displaystyle C_2^* = \frac{55}{9} \approx 6.11$}
\end{align*}

\textbf{Comparison with no-\( G \) case:}
\[
C_1^* = C_2^* = \frac{65}{9} \approx 7.22 \quad \Rightarrow \quad \text{Fall in consumption: } \frac{10}{9} \approx 1.11
\]

\vspace{1em}
\textbf{Current Account:} (using \( CA_t = Y_t - C_t - G_t + r B_{t-1} \), with \( B_0 = 0 \))

\textbf{Period 1:}
\begin{align*}
CA_1 &= (Y_1 - G_1) - C_1^* \\
CA_1 &= 3 - \frac{55}{9} \\
CA_1 &= -\frac{28}{9} \approx -3.11 \\
&\fcolorbox{red}{white}{$\displaystyle CA_1 = -\frac{28}{9} \approx -3.11$}
\end{align*}

\vspace{1em}
\textbf{Period 2:}
\begin{align*}
CA_2 &= (Y_2 - G_2) - C_2^* + r B_1 \\
CA_2 &= 10 - \frac{55}{9} + 0.25 \cdot \left(-\frac{28}{9} \right) \\
CA_2 &= \frac{35}{9} - \frac{7}{9} = \frac{28}{9} \approx 3.11 \\
&\fcolorbox{red}{white}{$\displaystyle CA_2 = \frac{28}{9} \approx 3.11$}
\end{align*}

\textit{Note: } \( CA_2 = -CA_1 \), as required by the condition \( B_2 = B_0 = 0 \).

\doublespacing

\textbf{A temporary increase in government spending} in period 1 reduces the \textbf{resources available for private consumption} today, shifting the \textbf{intertemporal budget constraint inward}. Since \textbf{households prefer to smooth consumption over time}, they “spread the pain” by lowering both \textbf{current and future consumption} rather than absorbing the entire adjustment in period 1.

As a result, the economy runs a \textbf{current account deficit} in the first period, \textbf{borrowing from abroad} to finance part of the government’s higher spending, and later repays through a \textbf{current account surplus} in the second period. This ensures that the \textbf{intertemporal budget constraint holds} while keeping \textbf{consumption as smooth as possible}.


\newpage

\section*{\noindent\textbf{Problem 2}}
\addcontentsline{toc}{section}{Problem 2}

Consider a two-period small open endowment economy. Suppose that the representative agent has lifetime utility given by:

\[
U = u(C_1) + \beta u(C_2),
\]

where \(\mathbf{0 < \beta < 1}\) is the subjective discount factor. Suppose the agent receives an endowment \(\mathbf{Y_1 = 1}\) and \(\mathbf{Y_2 = x}\). Assume that government consumes the amount \(\mathbf{G_1 \geq 0}\) and \(\mathbf{G_2 \geq 0}\) but balances its budget every period.

\subsection*{\noindent\textbf{a)}}
\addcontentsline{toc}{subsection}{a)}

Derive the intertemporal budget constraint and the current account for this economy
(assume that the economy’s initial net foreign assets are zero).

\vspace{0.5em}
\noindent\textcolor{formalred}{\textbf{ANSWER:}}

\singlespacing

% Requires: \usepackage{amsmath,xcolor}
\textbf{Assumptions:}
\begin{itemize}
  \item \( C_t \): consumption in period \( t \), \( t = 1, 2 \)
  \item \( Y_1 = 1,\; Y_2 = x \): endowment income
  \item \( G_t \): government spending in each period
  \item \( T_t = G_t \): Balanced Budget
  \item \( B_t \): net foreign asset position at end of period \( t \)
  \item \( r \): world interest rate
  \item \( \beta \in (0,1) \): discount factor
\end{itemize}

\textbf{Initial and Terminal Conditions}

\[
B_0 = 0 \quad \text{(initial NFA)}
\]

\[
B_2 = 0 \quad \text{(terminal NFA)}
\]

\textbf{Period 1 Budget Constraint}

\[
C_1 + B_1 = Y_1 - T_1
\]

\[
C_1 + B_1 = 1 - G_1 \quad (1)
\]

\textbf{Period 2 Budget Constraint}

\[
C_2 + B_2 = Y_2 - T_2 + (1+r)B_1
\]

\[
C_2 = x - G_2 + (1+r)B_1 \quad (2)
\]

\textbf{Substitute \( B_1 \) from (1):}

\[
B_1 = 1 - G_1 - C_1 \quad (3)
\]

\textbf{Into (2):}

\[
C_2 = x - G_2 + (1+r)(1 - G_1 - C_1) \quad (4)
\]

\textbf{Expand (4):}

\[
C_2 = x - G_2 + (1+r)(1 - G_1) - (1+r)C_1 \quad (5)
\]

\textbf{Rearrange (5):}

\[
(1+r)C_1 + C_2 = (1+r)(1 - G_1) + x - G_2 \quad (6)
\]

\textbf{Divide both sides by \( (1+r) \):}

\[
C_1 + \frac{C_2}{1+r} = (1 - G_1) + \frac{x - G_2}{1+r} \quad (7)
\]

\textbf{Intertemporal Budget Constraint}

\[
\fcolorbox{red}{white}{$
\displaystyle
C_1 + \frac{C_2}{1+r} = (1 - G_1) + \frac{x - G_2}{1+r}
$} \quad \text{[Intertemporal BC]}
\]

\textbf{Current Account — Period 1}

\[
CA_1 = Y_1 - G_1 - C_1 \quad (8)
\]

\[
CA_1 = 1 - G_1 - C_1 \quad (9)
\]

\[
\fcolorbox{red}{white}{$
\displaystyle
CA_1 = 1 - G_1 - C_1
$}
\]

\textbf{Current Account — Period 2}

\[
CA_2 = Y_2 - G_2 - C_2 \quad (10)
\]

\[
CA_2 = x - G_2 - C_2 \quad (11)
\]

\[
\fcolorbox{red}{white}{$
\displaystyle
CA_2 = x - G_2 - C_2
$}
\]

\subsection*{\noindent\textbf{b)}}
\addcontentsline{toc}{subsection}{b)}

If the period utility function is given by 
\[
u(C) = \frac{C^{1-\tfrac{1}{\sigma}}}{1 - \tfrac{1}{\sigma}}, 
\quad \text{where } \sigma > 0, \; \sigma \neq 1,
\]
derive the Euler equation for this economy. What is the autarky interest rate and how does it determine whether the economy runs a current account surplus or deficit?


\vspace{0.5em}
\noindent\textcolor{formalred}{\textbf{ANSWER:}}

\textbf{Given:} \(
u(C)=\dfrac{C^{\,1-\tfrac{1}{\sigma}}}{1-\tfrac{1}{\sigma}},\;
\sigma>0,\; \sigma \neq 1
\)

\textbf{Maximization Problem}

Maximize: \(
U = u(C_1) + \beta u(C_2)
\)

Subject to (IBC): \(
C_1 + \dfrac{C_2}{1+r} = (1-G_1) + \dfrac{x-G_2}{1+r}
\)

\textbf{Lagrangian}

\[
L = u(C_1) + \beta u(C_2) + \lambda\!\left[(1-G_1) + \frac{x-G_2}{1+r} - C_1 - \frac{C_2}{1+r}\right]
\]

\textbf{First-Order Conditions}

\[
\frac{\partial L}{\partial C_1} = u'(C_1) - \lambda = 0 \quad\text{(FOC1)}, \qquad
\frac{\partial L}{\partial C_2} = \beta u'(C_2) - \lambda\frac{1}{1+r} = 0 \quad\text{(FOC2)}
\]

From FOC1–FOC2: \(u'(C_1) = \beta(1+r)u'(C_2)\).

\textbf{Marginal Utility}

\[
u'(C) = C^{-\tfrac{1}{\sigma}}
\]

\textbf{Euler Equation}

\[
\fcolorbox{red}{white}{$
\displaystyle
C_1^{-\tfrac{1}{\sigma}} = \beta(1+r)\, C_2^{-\tfrac{1}{\sigma}}
$}
\]

Equivalently,
\[
\left(\frac{C_2}{C_1}\right)^{\tfrac{1}{\sigma}} = \beta(1+r)
\quad\Rightarrow\quad
\frac{C_2}{C_1} = [\beta(1+r)]^{\sigma}.
\]

\textbf{Autarky Interest Rate}

Autarky: \(B_1=0 \Rightarrow C_1 = 1-G_1,\; C_2 = x-G_2\).

Plug into Euler:
\[
(1-G_1)^{-\tfrac{1}{\sigma}} = \beta(1+r_a)(x-G_2)^{-\tfrac{1}{\sigma}}
\quad\Rightarrow\quad
\beta(1+r_a) = \left(\frac{x-G_2}{\,1-G_1\,}\right)^{\tfrac{1}{\sigma}}.
\]

\[
\fcolorbox{red}{white}{$
\displaystyle
r_a = \frac{1}{\beta}\left(\frac{x-G_2}{\,1-G_1\,}\right)^{\tfrac{1}{\sigma}} - 1
$}
\]

\textbf{CA Sign Condition}

If \( r < r_a \):

\[
[\beta(1+r)]^{\sigma} < \frac{x-G_2}{1-G_1}
\]

\[
\implies \; \frac{C_2}{C_1} < \frac{x-G_2}{1-G_1}
\]

\[
\implies \; C_1 > 1-G_1
\]

\[
\implies \; CA_1 = 1-G_1 - C_1 < 0 \quad \text{(Deficit)}
\]


If \( r > r_a \):

\[
[\beta(1+r)]^{\sigma} > \frac{x-G_2}{1-G_1}
\]

\[
\implies \; \frac{C_2}{C_1} > \frac{x-G_2}{1-G_1}
\]

\[
\implies \; C_1 < 1-G_1
\]

\[
\implies \; CA_1 = 1-G_1 - C_1 > 0 \quad \text{(Surplus)}
\]

\begin{itemize}
    \item When \( r < r_a \): the agent \textbf{borrows} to consume more today, causing a \textbf{CA deficit in period 1} and a \textbf{CA surplus in period 2}.
    
    \item When \( r > r_a \): the agent \textbf{saves}, consuming less today, leading to a \textbf{CA surplus in period 1} and a \textbf{CA deficit in period 2}.
    
    \item In both cases, the intertemporal budget constraint ensures the \textbf{net foreign asset position returns to zero} by the end.
    
    \item If \( r = r_a \): there is no incentive to trade intertemporally, so \textbf{consumption equals income in each period} and \(\textbf{CA = 0}\) throughout.
\end{itemize}


\subsection*{\noindent\textbf{c)}}
\addcontentsline{toc}{subsection}{c)}

Assume that \( \beta(1+r) = 1 \). Using the autarky interest rate derived in (b), show whether the country runs a current account surplus or deficit in period 1 when:  

\begin{enumerate}
    \item \( G_1 = G_2 = 0 \), and \( x < 1 \).
    \item \( G_1 > 0, \; G_2 = 0 \), and \( x = 1 \).
\end{enumerate}

\vspace{0.5em}
\noindent\textcolor{formalred}{\textbf{ANSWER (1):}}

\textbf{Recall: Autarky Interest Rate \( r_a \)}

From earlier:

\[
1+r_a = \frac{1}{\beta} \left( \frac{x-G_2}{1-G_1} \right)^{\tfrac{1}{\sigma}}
\]

Plug in \( G_1 = G_2 = 0 \):

\[
1+r_a = \frac{1}{\beta} \left( \frac{x}{1} \right)^{\tfrac{1}{\sigma}}
= \frac{1}{\beta} \, x^{\tfrac{1}{\sigma}}
\]

\textbf{Compare with World Interest Rate \( r \)}

Given:

\[
\beta(1+r) = 1 \quad \Rightarrow \quad 1+r = \frac{1}{\beta}
\]

Compare:

\[
1+r_a = \frac{1}{\beta} x^{\tfrac{1}{\sigma}}
\quad \text{vs.} \quad
1+r = \frac{1}{\beta}
\]

Since \( x < 1 \) and \( \sigma > 0 \), we get:

\[
x^{\tfrac{1}{\sigma}} < 1
\]

\[
\implies \; 1+r_a < 1+r
\]

\[
\implies \; r_a < r
\]


\textbf{Interpretation}

\[
r > r_a \;\; \Rightarrow \;\; \text{agent saves in period 1}
\]

\[
\Rightarrow \;\; C_1 < Y_1
\]

\[
\Rightarrow \;\; CA_1 > 0
\]

\textbf{Conclusion:} The country runs a \textbf{current account surplus in period 1}.

\begin{itemize}
    \item Given \( \beta(1+r) = 1 \), the consumer is \textbf{indifferent between consuming today or tomorrow} only if income is equal across periods.
    
    \item Since \( x < 1 \), \textbf{future income is lower}, so the autarky interest rate satisfies \( r_a < r \), making \textbf{saving attractive}.
    
    \item This leads the agent to \textbf{consume less than income today}, generating a \textbf{current account surplus in period 1}.
\end{itemize}

\vspace{0.5em}
\noindent\textcolor{formalred}{\textbf{ANSWER (2):}}

\textbf{Autarky Interest Rate}

Recall:

\[
1+r_a = \frac{1}{\beta} \left( \frac{x-G_2}{1-G_1} \right)^{\tfrac{1}{\sigma}}
\]

Substitute \( x=1, \; G_2=0 \):

\[
1+r_a = \frac{1}{\beta} \left( \frac{1}{1-G_1} \right)^{\tfrac{1}{\sigma}}
\]

Since \( G_1 > 0 \), then 

\[
\frac{1}{1-G_1} > 1
\]

\[
\implies \left( \frac{1}{1-G_1} \right)^{\tfrac{1}{\sigma}} > 1
\]

\[
\implies 1+r_a > \frac{1}{\beta} = 1+r
\]

\[
\implies r_a > r
\]

\textbf{Interpretation}

\[
r_a > r
\]

\[
\implies \text{agent prefers to \textbf{borrow} (consume more today)}
\]

\[
\implies C_1 > Y_1 - G_1
\]

\[
\implies CA_1 < 0
\]

\textbf{Conclusion:} The country runs a \textbf{current account deficit in period 1}.

\begin{itemize}
    \item With \( x = 1 \), income is symmetric across periods.  
    \item Since \( G_1 > 0 \), \textbf{net resources are lower today}.  
    \item This makes \textbf{future resources relatively more attractive}, so \( r_a > r \).  
    \item The agent \textbf{borrows to smooth consumption} "spreading the pain".  
    \item Thus, \textbf{consumption exceeds income in period 1}, leading to a \textbf{current account deficit}.
\end{itemize}


\section*{\noindent\textbf{Problem 3}}
\addcontentsline{toc}{section}{Problem 3}

\doublespacing

Consider a two-period small open endowment economy. Assume that the period utility 
function is given by 
\[
u(C) = \frac{C^{\,1-\tfrac{1}{\sigma}}}{1 - \tfrac{1}{\sigma}}, 
\quad \sigma > 0, \; \sigma \neq 1.
\] 

Further assume that the world interest rate \( r \) is set at such a level such that the small open economy 
is a net borrower \( b_1 \) in period 1. 

Now suppose that the government introduces capital controls thereby restricting the amount 
the small open economy can borrow from the rest of the world. Let \( \bar{b}_1 \) denote the 
level of borrowing in the presence of capital controls.

\subsection*{\noindent\textbf{a)}}
\addcontentsline{toc}{subsection}{a)}

Show for which values of debt will these capital controls affect equilibrium consumption in the small open economy.

\vspace{0.5em}
\noindent\textcolor{formalred}{\textbf{ANSWER:}}

\singlespacing
\textbf{Assumptions:}
\begin{itemize}
    \item Two-period small open economy
    \item Utility: \( u(C) = \dfrac{C^{\,1-\tfrac{1}{\sigma}}}{1 - \tfrac{1}{\sigma}}, \quad \sigma > 0, \; \sigma \neq 1 \)
    \item World interest rate \( r \)
    \item Initial borrowing without restrictions: \( b_1^* \)
    \item Capital controls impose: \( b_1 \leq \bar{b}_1 \)
\end{itemize}

\textbf{Maximizing Problem}

Maximize:  
\[
U = u(C_1) + \beta u(C_2)
\]

Subject to (Recall IBC):  
\[
C_1 + \frac{C_2}{1+r} = Y_1 + \frac{Y_2}{1+r} \quad (1)
\]

\textbf{Euler Equation}

\[
C_1^{-\tfrac{1}{\sigma}} = \beta(1+r)\, C_2^{-\tfrac{1}{\sigma}} \quad (2)
\]

\[
\frac{C_2}{C_1} = [\beta(1+r)]^{\sigma} \quad (3)
\]

\textbf{Substitute into (1)}

\[
C_1 + \frac{[\beta(1+r)]^{\sigma} \, C_1}{1+r} = Y_1 + \frac{Y_2}{1+r} \quad (4)
\]

\[
C_1 \left[ 1 + \frac{[\beta(1+r)]^{\sigma}}{1+r} \right] = Y_1 + \frac{Y_2}{1+r} \quad (5)
\]

\textbf{Optimal Consumption}

\[
C_1^* = \frac{Y_1 + \tfrac{Y_2}{1+r}}{1 + \tfrac{[\beta(1+r)]^{\sigma}}{1+r}} \quad (6)
\]

\textbf{Optimal Borrowing}

\[
b_1^* = Y_1 - C_1^* \quad (7)
\]

\textbf{Capital Control Constraint}

If  
\[
b_1^* > \bar{b}_1 \quad (8)
\]  
then the constraint binds.  

New consumption under controls:  

\[
C_1 = Y_1 - \bar{b}_1 \quad (9)
\]

\[
C_2 = Y_2 + (1+r)\bar{b}_1 \quad (10)
\]

\textbf{Conclusion:}

\[
\fcolorbox{red}{white}{$
\; \text{Capital controls affect equilibrium consumption iff } b_1^* > \bar{b}_1 \;
$}
\]


\begin{itemize}
    \item Without controls, the agent chooses optimal borrowing \( b_1^* \) to \textbf{smooth consumption over time}.
    
    \item If the government imposes a tighter cap \( \bar{b}_1 < b_1^* \), the agent \textit{cannot borrow as much} and must \textbf{reduce consumption in period 1}.
    
    \item This constraint \textbf{breaks optimal smoothing}, so equilibrium consumption changes only when controls \textbf{bind} — that is, when \( b_1^* > \bar{b}_1 \).
\end{itemize}


\subsection*{\noindent\textbf{b)}}
\addcontentsline{toc}{subsection}{b)}

What will the interest rate be in the presence of capital controls?

\vspace{0.5em}
\noindent\textcolor{formalred}{\textbf{ANSWER:}}


\textbf{Assume: Capital Controls Bind}  

\[
b_1 = \bar{b}_1 \quad (1)
\]

\textbf{Period 1 Consumption}  

\[
C_1 = Y_1 - \bar{b}_1 \quad (2)
\]

\textbf{Period 2 Consumption}  

\[
C_2 = Y_2 + (1+r)\bar{b}_1 \quad (3)
\]

\textbf{Euler Equation (Always Holds):}  

\[
C_1^{-\tfrac{1}{\sigma}} = \beta(1+r) \, C_2^{-\tfrac{1}{\sigma}} \quad (4)
\]

\textbf{Substitute (2) and (3) into (4):}  

\[
(Y_1 - \bar{b}_1)^{-\tfrac{1}{\sigma}}
= \beta(1+r) \, \big(Y_2 + (1+r)\bar{b}_1\big)^{-\tfrac{1}{\sigma}} \quad (5)
\]

\textbf{Rearrange:}  

\[
\frac{(Y_2 + (1+r)\bar{b}_1)^{-\tfrac{1}{\sigma}}}{(Y_1 - \bar{b}_1)^{-\tfrac{1}{\sigma}}}
= \beta(1+r) \quad (6)
\]

\textbf{Raise both sides to power \( -\sigma \):}  

\[
\frac{Y_2 + (1+r)\bar{b}_1}{Y_1 - \bar{b}_1}
= \big[\beta(1+r)\big]^{\sigma} \quad (7)
\]

\textbf{Final Expression (Implicit Equation for \(r\)):}  

\[
\fcolorbox{red}{white}{$
\displaystyle 
\frac{Y_2 + (1+r)\bar{b}_1}{\,Y_1 - \bar{b}_1\,} = \big[\beta(1+r)\big]^{\sigma}
$}
\]

This is the \textbf{implicit equation for \(r\)} in the presence of binding capital controls.

\begin{itemize}
    \item \( r \) \textbf{adjusts} so that the agent, forced to borrow only \( \bar{b}_1 \), is still \textbf{optimizing consumption}.
    
    \item It \textbf{balances} how much "extra future consumption*"the agent gets from borrowing today, given the constraint.
    
    \item Thus, it is the \textbf{interest rate that makes the Euler equation hold} under the imposed borrowing limit.
\end{itemize}


\subsection*{\noindent\textbf{c)}}
\addcontentsline{toc}{subsection}{c)}

Analyze the effect of a temporary increase in the endowment of the economy. Will
consumption smoothing take place in the presence of capital controls?

\vspace{0.5em}
\noindent\textcolor{formalred}{\textbf{ANSWER:}}

\textbf{Define the Shock}

Only in Period 1  

Original: \( Y_1, Y_2 \)  

Shock:  
\[
Y_1' = Y_1 + \Delta, \quad \Delta > 0, 
\qquad Y_2' = Y_2
\]

\textbf{Unconstrained Optimum}

With no capital controls, the agent solves:  
\[
C_1 + \frac{C_2}{1+r} = Y_1' + \frac{Y_2'}{1+r}
\]

Smoothing condition (Euler equation):  
\[
C_1^{-\tfrac{1}{\sigma}} = \beta(1+r)\, C_2^{-\tfrac{1}{\sigma}}
\]

\(\Rightarrow\) Agent \textbf{spreads the shock \(\Delta\) across both periods}.

\textbf{With Capital Controls}

Let \(\bar{b}_1\) be binding before the shock.  

New income: \( Y_1' = Y_1 + \Delta \)  

Borrowing limit unchanged: \( b_1 = \bar{b}_1 \)  

Then:  
\[
C_1 = Y_1' - \bar{b}_1 = (Y_1 + \Delta) - \bar{b}_1
\]

\[
C_2 = Y_2 + (1+r)\bar{b}_1
\]

\(\Rightarrow\) \textbf{Consumption rises only in period 1}.  

\(\Rightarrow\) \textbf{No adjustment in period 2}.  

\textbf {Conclusion}

\[
\fcolorbox{red}{white}{$
\; \text{No consumption smoothing under capital controls if } \bar{b}_1 \text{ still binds after the shock.} \;
$}
\]

\begin{itemize}
    \item With capital controls binding, the agent \textbf{cannot borrow or lend freely} to reallocate income over time.  
    
    \item When income \textbf{rises temporarily in period 1}, they \textbf{cannot save the extra} to shift consumption to period 2.  
    
    \item As a result, they are \textbf{forced to consume the entire increase immediately}, which \textbf{breaks consumption smoothing}.
\end{itemize}
 

\section*{\noindent\textbf{Problem 4}}
\addcontentsline{toc}{section}{Problem 4}

Consider a small open production economy with an infinite horizon. Suppose the production 
technology is given by 
\[
Y_t = A_t F(K_t)
\]
where \( A_t \) denotes a positive productivity parameter and \( F(K_t) \) is a strictly increasing and strictly concave function.  

Capital accumulates according to 
\[
K_{t+1} = K_t + I_t
\]
and net foreign assets accumulate via:
\[
B_{t+1} = Y_t + (1+r)B_t - C_t - I_t.
\]


\subsection*{\noindent\textbf{a)}}
\addcontentsline{toc}{subsection}{a)}

Derive the first-order conditions for this problem

\vspace{0.5em}
\noindent\textcolor{formalred}{\textbf{ANSWER:}}

\textbf{Maximizing Problem}  

\[
\max_{\{C_t, K_{t+1}, B_{t+1}\}} \sum_{s=t}^{\infty} \beta^{s-t} u(C_t)
\]

\textbf{Subject to}  

\[
B_{t+1} = A_t F(K_t) + (1+r)B_t - C_t - I_t \quad (1)
\]

\[
K_{t+1} = K_t + I_t
\]

\[
\implies\; I_t = K_{t+1} - K_t \quad (2)


\textbf{Substitute (2) into (1)}  

\[
B_{t+1} = A_t F(K_t) + (1+r)B_t - C_t - (K_{t+1} - K_t) \quad (3)
\]

\[
B_{t+1} = A_t F(K_t) + (1+r)B_t - C_t - K_{t+1} + K_t \quad (4)
\]

\textbf{Lagrangian}  

\[
L = \sum_{t=0}^{\infty} \beta^t 
\Big[ u(C_t) + \lambda_t \big( A_t F(K_t) + (1+r)B_t - C_t - K_{t+1} + K_t - B_{t+1} \big) \Big] 
\quad (5)
\]

\textbf{First-Order Condition  \(C_t\)}  

\[
\frac{\partial L}{\partial C_t} = \beta^t \big[u'(C_t) - \lambda_t \big] = 0
\]

\[
\implies \lambda_t = u'(C_t) \quad (6)
\]

\textbf{First-Order Condition  \(B_{t+1}\)}  

\[
\frac{\partial L}{\partial B_{t+1}} = -\beta^t \lambda_t + \beta^{t+1} \lambda_{t+1}(1+r) = 0
\]

\[
\implies \lambda_t = \beta(1+r)\lambda_{t+1} \quad (7)
\]

\textbf{First-Order Condition  \(K_{t+1}\)}  

\[
\frac{\partial L}{\partial K_{t+1}} = -\beta^t \lambda_t 
+ \beta^{t+1}\lambda_{t+1}\big(A_{t+1}F'(K_{t+1})+1\big) = 0
\]

\[
\implies \lambda_t = \beta \lambda_{t+1}\big(A_{t+1}F'(K_{t+1})+1\big) \quad (8)
\]

\textbf{Final First-Order Conditions}

\[
\fcolorbox{red}{white}{$
\begin{aligned}
(1) \quad & \lambda_t = u'(C_t) \\[6pt]
(2) \quad & \lambda_t = \beta(1+r)\lambda_{t+1} \\[6pt]
(3) \quad & \lambda_t = \beta \lambda_{t+1}\big(A_{t+1}F'(K_{t+1})+1\big)
\end{aligned}
$}
\]

\subsection*{\noindent\textbf{b)}}
\addcontentsline{toc}{subsection}{b)}

Derive the intertemporal budget constraint and TVC condition.

\vspace{0.5em}
\noindent\textcolor{formalred}{\textbf{ANSWER:}}

\textbf{Budget Constraint}

\[
B_{t+1} = A_t F(K_t) + (1+r)B_t - C_t - I_t \quad (1)
\]

\textbf{From capital accumulation}

\[
K_{t+1} = K_t + I_t 
\]
\[
\implies\; I_t = K_{t+1} - K_t \quad (2)
\]

\textbf{Substitute (2) into (1)}

\[
B_{t+1} = A_t F(K_t) + (1+r)B_t - C_t - (K_{t+1}-K_t) \quad (3)
\]

\[
B_{t+1} = A_t F(K_t) + (1+r)B_t - C_t - K_{t+1} + K_t \quad (4)
\]

\textbf{Solve forward for \(B_{t+1}\)}

Start with (1):
\[
B_{t+1} = A_t F(K_t) + (1+r)B_t - C_t - I_t
\]

Divide both sides by \((1+r)^{t+1}\):
\[
\frac{B_{t+1}}{(1+r)^{t+1}}
= \frac{A_t F(K_t)}{(1+r)^{t+1}}
+ \frac{B_t}{(1+r)^t}
- \frac{C_t}{(1+r)^{t+1}}
- \frac{I_t}{(1+r)^{t+1}} \quad (5)
\]

Repeat forward up to \(T\), then sum both sides:
\[
\frac{B_{T+1}}{(1+r)^{T+1}}
= \frac{B_t}{(1+r)^t}
+ \sum_{s=t}^{T} \frac{A_s F(K_s) - C_s - I_s}{(1+r)^{s+1}}
\quad (6)
\]

\textbf{Intertemporal Budget Constraint (IBC)}

Let \(T \to \infty\). Then
\[
\lim_{T\to\infty} \frac{B_{T+1}}{(1+r)^{T+1}} = 0
\quad \text{(TVC)} \quad (7)
\]

Thus,
\[
\frac{B_t}{(1+r)^t}
= \sum_{s=t}^{\infty} \frac{C_s + I_s - A_s F(K_s)}{(1+r)^{s+1}}
\quad \text{(IBC)}
\]

Or equivalently:
\[
\sum_{s=t}^{\infty} \frac{C_s + I_s}{(1+r)^{s+1}}
= \sum_{s=t}^{\infty} \frac{A_s F(K_s)}{(1+r)^{s+1}}
+ \frac{B_t}{(1+r)^t}
\]

\textbf{Transversality Condition (TVC)}

\[
\lim_{t\to\infty} \frac{B_t}{(1+r)^t} = 0
\]

This rules out Ponzi schemes (indefinitely rolling over debt).


\subsection*{\noindent\textbf{c)}}
\addcontentsline{toc}{subsection}{c)}

Assuming \((1+r)\beta = 1\) where \(0 < \beta < 1\) is the subjective discount factor, characterize the perfect foresight path for a constant value of \(A_t\) (i.e., \(A_t = A\) for all \(t = 0,1,2,\ldots\)).

\vspace{0.5em}
\noindent\textcolor{formalred}{\textbf{ANSWER:}}

\textbf{Given}  

\[
(1+r)\beta = 1
\]

\[
A_t = A \quad \forall t \quad \text{(constant productivity)}
\]

\textbf{Consumption Euler Equation}  

\[
u'(C_t) = \beta(1+r)u'(C_{t+1}) \quad (1)
\]

Since \(\beta(1+r)=1\):  

\[
u'(C_t) = u'(C_{t+1}) \quad (2)
\]

\textbf{Implication: Constant Consumption}  

\[
C_t = C_{t+1} = C_{t+2} = \cdots \quad \Rightarrow \quad C_t = \bar{C} \quad (3)
\]

\textbf{Investment Euler Condition}  

\[
F'(K_{t+1}) = r A \quad (4)
\]

With \(A_t=A\):  

\[
F'(K_{t+1}) = r \quad \forall t \quad (5)
\]

\textbf{Capital Stock Constant}  

\[
K_{t+1} = \bar{K} \quad (6)
\]

\[
I_t = K_{t+1} - K_t = 0 \quad (7)
\]

\textbf{Output Constant}  

\[
Y_t = AF(K_t) = AF(\bar{K}) = \bar{Y} \quad (8)
\]

\textbf{Budget Constraint in Steady State}  

\[
B_{t+1} = AF(\bar{K}) + (1+r)B_t - \bar{C} \quad (9)
\]

\[
B_{t+1} = (1+r)B_t + \bar{Y} - \bar{C} \quad (10)
\]

\textbf{Steady State Condition (No Debt Explosion)}  

\[
\lim_{t \to \infty} \frac{B_t}{(1+r)^t} = 0 
\quad \implies \quad \bar{C} = \bar{Y} \quad (11)
\]

Then:  

\[
B_{t+1} = (1+r)B_t + \bar{Y} - \bar{Y} = (1+r)B_t 
\quad \implies \quad B_t = 0 \quad (12)
\]

(assuming initial \(B_0 = 0\))  

\textbf{ Perfect Foresight Path}  

\[
\fcolorbox{red}{white}{$
\begin{aligned}
C_t &= \bar{C} = AF(\bar{K}) \\[6pt]
K_t &= \bar{K} \quad \text{such that } F'(\bar{K}) = r \\[6pt]
I_t &= 0 \\[6pt]
B_t &= 0
\end{aligned}
$}
\]

Perfect foresight path: \textbf{no growth, constant consumption, efficient capital allocation}.


\subsection*{\noindent\textbf{d)}}
\addcontentsline{toc}{subsection}{d)}

Suppose that in period \( t = -1 \) the economy is in the equilibrium derived in part (c) above. Suppose that in period 0 a temporary positive productivity shock occurs such that \( A \) rises from \( A \) to \( A^* \), and then returns to its initial level in period 1. Show that this results in an increase in consumption and period 0 saving, an initial improvement in the trade balance, and a current account surplus.

\vspace{0.5em}
\noindent\textcolor{formalred}{\textbf{ANSWER:}}

\textbf{Assumptions:}
\begin{itemize}
    \item At \(t=-1\), the economy is in steady state:  
    \[
    A_{-1} = A, \quad C_t = \bar{C}, \quad K_t = \bar{K}, \quad B_t = 0
    \]
    \item At \(t=0\), a temporary productivity shock occurs:  
    \[
    A_0 = A^* > A, \quad A_1 = A
    \]
\end{itemize}

\textbf{Effect on Output}  

\[
Y_0 = A^*F(K_0) > AF(K_0) = \bar{Y} \quad (1)
\]

\(\Rightarrow\) Output is temporarily higher in period 0.  

\textbf{Household Behavior}  

Euler equation:  
\[
u'(C_0) = \beta(1+r)u'(C_1) \quad (2)
\]

Since \(C_1 = \bar{C}\), the household smooths consumption:  
\[
C_0 \uparrow, \quad C_0 < Y_0 \quad \implies \quad \text{saves the difference} \quad (3)
\]

\textbf{Period 0 Saving}  

Budget constraint:  
\[
B_1 = Y_0 + (1+r)B_0 - C_0 - I_0 \quad (4)
\]

In steady state: \(B_0=0, \; I_0=0, \; K_1=K_0=\bar{K}\).  

So:  
\[
B_1 = Y_0 - C_0 > 0 \quad (5)
\]

\(\Rightarrow\) The agent accumulates foreign assets (\textbf{saves}).  

\textbf{Trade Balance and Current Account}  

Trade balance:  
\[
TB_0 = Y_0 - C_0 - I_0
\]

Since \(I_0=0\) and \(Y_0 > C_0\):  
\[
TB_0 > 0 \quad \text{(trade balance improves)}
\]

Current account:  
\[
CA_0 = TB_0 + rB_0 = TB_0 + 0 = TB_0 \quad \implies \quad CA_0 > 0
\]

\textbf{Results of a Temporary Productivity Shock in t}  

\[
\fcolorbox{red}{white}{$
\begin{aligned}
C_0 &\uparrow \quad \text{(consumption increases)} \\[6pt]
B_1 &> 0 \quad \text{(period 0 saving)} \\[6pt]
TB_0 &> 0 \quad \text{(trade balance improves)} \\[6pt]
CA_0 &> 0 \quad \text{(current account surplus)}
\end{aligned}
$}
\]



\end{document}
