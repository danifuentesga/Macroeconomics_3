\documentclass[12pt]{article}

% --- Paquetes ---
\usepackage{pifont} 
\usepackage{tikz}
\usepackage{pgfplots}
\pgfplotsset{compat=1.18}
\usepackage[most]{tcolorbox}
\usepackage[spanish,es-tabla]{babel}   % español
\usepackage[utf8]{inputenc}            % acentos
\usepackage[T1]{fontenc}
\usepackage{lmodern}
\usepackage{geometry}
\usepackage{fancyhdr}
\usepackage{xcolor}
\usepackage{titlesec}
\usepackage{lastpage}
\usepackage{amsmath,amssymb}
\usepackage{enumitem}
\usepackage[table]{xcolor} % para \cellcolor y \rowcolor
\usepackage{colortbl}      % colores en tablas
\usepackage{float}         % para usar [H] si quieres fijar la tabla
\usepackage{array}         % mejor control de columnas
\usepackage{amssymb}       % para palomita
\usepackage{graphicx}      % para logo github
\usepackage{hyperref}
\usepackage{setspace} % para hipervinculo
\usepackage[normalem]{ulem}
\usepackage{siunitx}       % Asegúrate de tener este paquete en el preámbulo
\usepackage{booktabs}
\sisetup{
    output-decimal-marker = {.},
    group-separator = {,},
    group-minimum-digits = 4,
    detect-all
}

% Etiqueta en el caption (en la tabla misma)
\usepackage{caption}
\captionsetup[table]{name=Tabla, labelfont=bf, labelsep=period}

% Prefijo en la *Lista de tablas*
\usepackage{tocloft}
\renewcommand{\cfttabpresnum}{Tabla~} % texto antes del número
\renewcommand{\cfttabaftersnum}{.}    % punto después del número
\setlength{\cfttabnumwidth}{5em}      % ancho para "Tabla 10." ajusta si hace falta



% --- Márgenes y encabezado ---
\geometry{left=1in, right=1in, top=1in, bottom=1in}

% Alturas del encabezado (un poco más por las 2–3 líneas del header)
\setlength{\headheight}{32pt}
\setlength{\headsep}{20pt}

\definecolor{maroon}{RGB}{128, 0, 0}

\pagestyle{fancy}
\fancyhf{}

% Regla del encabezado (opcional)
\renewcommand{\headrulewidth}{0.4pt}

% Encabezado izquierdo
\fancyhead[L]{%
  \textcolor{maroon}{\textbf{El Colegio de México}}\\
  \textbf{Macroeconomics 3}
}

% Encabezado derecho
\fancyhead[R]{%
  Topic 2B : An Introduction to Dynamic
Programming\\
  \textbf{Jose Daniel Fuentes García}\\
  Github : \includegraphics[height=1em]{github.png}~\href{https://github.com/danifuentesga}{\texttt{danifuentesga}}
}

% Número de página al centro del pie
\fancyfoot[C]{\thepage}

% --- APLICAR EL MISMO ESTILO A PÁGINAS "PLAIN" (TOC, LOT, LOF) ---
\fancypagestyle{plain}{%
  \fancyhf{}
  \renewcommand{\headrulewidth}{0.4pt}
  \fancyhead[L]{%
    \textcolor{maroon}{\textbf{El Colegio de México}}\\
    \textbf{Macroeconomics 3}
  }
  \fancyhead[R]{%
    Topic 2B : An Introduction to Dynamic
Programming\\
    \textbf{Jose Daniel Fuentes García}\\
    Github : \includegraphics[height=1em]{github.png}~\href{https://github.com/danifuentesga}{\texttt{danifuentesga}}
  }
  \fancyfoot[C]{\thepage}
}

% Pie de página centrado
\fancyfoot[C]{\thepage\ de \pageref{LastPage}}

\renewcommand{\headrulewidth}{0.4pt}

% --- Color principal ---
\definecolor{formalblue}{RGB}{0,51,102} % azul marino sobrio

% --- Estilo de títulos ---
\titleformat{\section}[hang]{\bfseries\Large\color{formalblue}}{}{0em}{}[\titlerule]
\titleformat{\subsection}{\bfseries\large\color{formalblue}}{\thesubsection}{1em}{}


% --- Listas ---
\setlist[itemize]{leftmargin=1.2em}

% --- Sin portada ---
\title{}
\author{}
\date{}

\begin{document}

\begin{titlepage}
    \vspace*{-1cm}
    \noindent
    \begin{minipage}[t]{0.49\textwidth}
        \includegraphics[height=2.2cm]{colmex.jpg}
    \end{minipage}%
    \begin{minipage}[t]{0.49\textwidth}
        \raggedleft
        \includegraphics[height=2.2cm]{cee.jpg}
    \end{minipage}

    \vspace*{2cm}

    \begin{center}
        \Huge \textbf{CENTRO DE ESTUDIOS ECONÓMICOS} \\[1.5em]
        \Large Maestría en Economía 2024--2026 \\[2em]
        \Large Macroeconomics 3 \\[3em]
        \LARGE \textbf{Topic 2B : An Introduction to Dynamic
Programming} \\[6em]
        \large \textbf{Disclaimer:} I AM NOT the original intellectual author of the material presented in these notes. The content is STRONGLY based on a combination of lecture notes (Stephen McKnight), textbook references, and personal annotations for learning purposes. Any errors or omissions are entirely my own responsibility.\\[0.9em]
        
    \end{center}

    \vfill
\end{titlepage}

\newpage

\setcounter{secnumdepth}{2}
\setcounter{tocdepth}{3}
\tableofcontents

\newpage

\section*{\noindent\textbf{2B.1 Introduction and Aims}}
\addcontentsline{toc}{section}{2B.1 Introduction and Aims}

\begin{itemize}
    \item This course introduces the \textbf{fundamentals} of \textit{dynamic programming theory}.
    \item We will focus exclusively on \textbf{deterministic dynamic programming}, assuming \textit{no uncertainty}.
    \item The material provides a \textbf{basic foundation} for solving \textit{macroeconomic models} using deterministic dynamic programming techniques.
    \item The main objective is to \textbf{equip you} with the tools to solve \textit{dynamic optimization problems} using this approach.
    \item The learning method is primarily \textbf{hands-on}, emphasizing \textit{learning by doing}.
    \item You will explore a \textbf{variety of solution strategies} applicable to solving \textit{recursive problems}.
    \item Keep in mind that \textbf{modern macroeconomic dynamic programming problems} often cannot be solved analytically and typically require \textit{numerical methods} and \textbf{software tools} like MATLAB—topics usually covered in graduate school.
\end{itemize}

\begin{center}
    \textcolor{red}{\textbf{\uline{Suggested Reading}}}
\end{center}

\begin{itemize}
    \item \textbf{The material of this class is based on:} \\
    \textit{Adda and Cooper (2003), Dynamic Economics: Quantitative Methods \& Applications, MIT Press, Chapter 2.}

    \item \textbf{There are other texts that also briefly cover dynamic programming which you may find useful. For example:} \\
    \textit{Ljungqvist and Sargent (2004), Recursive Macroeconomic Theory, MIT Press (2nd Edition), Chapters 1 and 3.}

    \item \textbf{For a PhD level coverage of the material we cover see:} \\
    \textit{Stokey and Lucas (1989), Recursive Methods in Economic Dynamics, Harvard University Press, Chapter 2.1 and Chapter 4.}
\end{itemize}

\subsection*{\noindent\textbf{Direct Method / Sequence Approach}}
\addcontentsline{toc}{subsection}{Direct Method / Sequence Approach}

\begin{enumerate}
    \item Euler equation \& other FOCs
    \item Lifetime budget constraint + TVC imposed
\end{enumerate}

\vspace{0.5cm}

\subsection*{\noindent\textbf{Dynamic Programming}}
\addcontentsline{toc}{subsection}{Dynamic Programming}

\begin{enumerate}
    \item[\textbf{A.}] \textbf{Euler Equation Method} – Here you don’t need to know the form of the value function, but it has to be differentiable. Same procedure as the Direct Method
    \item[\textbf{B.}] \textbf{Solve the Bellman Equation} – Here the \textit{value function does not need to be differentiable}, and you find the form of the value function. Either:
    \begin{enumerate}
        \item Value function iteration method
        \item Guess \& verify method
        \item Policy function iteration method
    \end{enumerate}
\end{enumerate}

\section*{\noindent\textbf{2B.2 An Overview of Dynamic Programming}}
\addcontentsline{toc}{section}{2B.2 An Overview of Dynamic Programming}

\begin{itemize}
    \item At its core, the dynamic programming approach builds on the concept of \textit{indirect utility}, which you may recognize from standard consumer theory.
    \item This method allows us to compactly represent the value of being in a certain situation or \textbf{state}.
    \item Dynamic programming extends this concept to environments that evolve over time.
    \item To connect with earlier knowledge, recall how we express indirect utility in static consumer theory.
    \item In that framework, consumers solve:
    \[
        V(I, p) = \max_{c} u(c),
    \]
    \item subject to the constraint \( pc = I \)
    \item where \( c \) is a vector of consumption goods, \( p \) is a vector of prices, and \( I \) is income.
\end{itemize}

\textbf{Maximization problem:}
\[
\max_{c} u(c) \quad \text{s.t.} \quad pc = I
\]

\textbf{Lagrangian:}
\[
\mathcal{L}(c, \lambda) = u(c) + \lambda (I - pc)
\]

\textbf{First-order condition:}
\[
\frac{\partial \mathcal{L}}{\partial c_j} = u_j(c) - \lambda p_j = 0
\Rightarrow \frac{u_j(c)}{p_j} = \lambda
\]

\textbf{Envelope theorem:}
\[
\frac{\partial V(I, p)}{\partial I} = \lambda
\Rightarrow \lambda = V_I(I, p)
\]

\textbf{Substitute:}
\[
\frac{u_j(c)}{p_j} = V_I(I, p)
\]

\begin{itemize}
    \item The first-order condition is:
    \[
    \frac{u_j(c)}{p_j} = \lambda
    \]
    \item for all \( j \), where \( \lambda \) is the multiplier on the budget constraint and \( u_j(c) \) is the marginal utility from good \( j \)
    \item Here \( V(I, p) \) is an indirect utility function: it gives you the maximized level of utility from the current state \( (I, p) \)
    \item If the consumer is in this state, you can predict that they will attain this level of utility
    \item You do not need to know what the consumer will do with their income: it is enough to know that they will act optimally
    \item This is very powerful logic and underlies the idea behind dynamic programming
    \item To illustrate, what happens if we give the consumer some more income?
    \item Well, welfare goes up by \( V_I(I, p) > 0 \). Can I predict what will happen when I give the consumer some more income?
    \item Not really, since the optimizing consumer is indifferent with respect to how this extra income is spent:
    \[
    \frac{u_j(c)}{p_j} = V_I(I, p)
    \]
    \item for all \( j \)
\end{itemize}

\begin{itemize}
    \item The indirect utility function helps summarize the solution to the consumer's optimization problem and reveals the marginal value of income, even without detailed knowledge of their consumption behavior.
    
    \item Consider a basic example with two goods, where the consumer maximizes:
    
    
    \[
    \max_{c_1, c_2} \left[ a \log c_1 + (1 - a) \log c_2 \right]
    \]
    subject to the budget constraint:
    \[
    p_1 c_1 + p_2 c_2 = I
    \]
    
    \item The first-order conditions for this problem are:
    \[
    p_1 c_1 = aI, \quad p_2 c_2 = (1 - a)I
    \]
    
    \item Solving for \( c_1 \) and \( c_2 \), and plugging them into the utility function, gives us the indirect utility:
    \[
    V(I, p_1, p_2) = a \log \left( \frac{a}{p_1} \right) + (1 - a) \log \left( \frac{1 - a}{p_2} \right) + \log I
    \]
    
    \item This shows how utility depends on the state variables \( I \), \( p_1 \), and \( p_2 \). 
    
    \item \textit{In dynamic programming terminology, this is called the value function}, because it maps the state variables to the maximum attainable value.
\end{itemize}

\textcolor{blue}{\textbf{\uline{Full procedure}}}

{\color{blue}
\textbf{Maximization problem:}
\[
\max_{c_1, c_2} \left[ a \log c_1 + (1 - a) \log c_2 \right]
\quad \text{s.t.} \quad p_1 c_1 + p_2 c_2 = I
\]

\textbf{Lagrangian:}
\[
\mathcal{L}(c_1, c_2, \lambda) = a \log c_1 + (1 - a) \log c_2 + \lambda (I - p_1 c_1 - p_2 c_2)
\]

\textbf{First-order conditions:}
\begin{align*}
\frac{\partial \mathcal{L}}{\partial c_1} &: \quad \frac{a}{c_1} - \lambda p_1 = 0
\Rightarrow p_1 c_1 = aI \\
\frac{\partial \mathcal{L}}{\partial c_2} &: \quad \frac{1 - a}{c_2} - \lambda p_2 = 0
\Rightarrow p_2 c_2 = (1 - a)I
\end{align*}

\textbf{Solve for } $c_1$ \textbf{ and } $c_2$:
\[
c_1 = \frac{aI}{p_1}, \quad c_2 = \frac{(1 - a)I}{p_2}
\]

\textbf{Substitute into utility:}
\[
V(I, p_1, p_2) = a \log \left( \frac{aI}{p_1} \right) + (1 - a) \log \left( \frac{(1 - a)I}{p_2} \right)
\]

\textbf{Simplify:}
\[
V(I, p_1, p_2) = a \log \left( \frac{a}{p_1} \right) + (1 - a) \log \left( \frac{1 - a}{p_2} \right) + \log I
\]
}

\section*{\noindent\textbf{2B.3 Finite-Horizon Dynamic Programming}}
\addcontentsline{toc}{section}{2B.3 Finite-Horizon Dynamic Programming}

\begin{itemize}
    \item We now study a basic dynamic optimization scenario: the classic \textit{cake-eating problem}.
    
    \item We begin by analyzing the case with a finite time horizon before moving to the infinite horizon case.
    
    \item Suppose you are initially endowed with a cake of size \( W_1 \).
    
    \item At every time period \( t = 1, 2, 3, \dots, T \), you can consume a portion of the cake, but must save the remaining amount for future periods.
    
    \item Let \( c_t \) denote consumption in period \( t \), and let \( u(c_t) \) represent the instantaneous utility derived from consuming \( c_t \).
    
    \item The utility function is assumed to be time-invariant — this reflects stationary preferences.
    
    \item Assume that \( u(\cdot) \) is strictly increasing and strictly concave. Additionally, it satisfies the Inada condition:
    \[
    \lim_{c \to 0} u'(c) = \infty
    \]
    
    \item The Inada condition ensures that consumption in every period is strictly positive — we always consume at least a small amount of cake.
    
    \item Therefore, the agent’s lifetime utility over \( T \) periods is:
    \[
    \sum_{t=1}^{T} \beta^{(t - 1)} u(c_t)
    \]
    
    \item where the discount factor \( \beta \) lies in the interval \( 0 < \beta < 1 \).
\end{itemize}

\begin{itemize}
    \item Assume that the cake does \textit{not depreciate} — it neither spoils nor grows over time. This means total cake only changes when the agent consumes it.

    \item The evolution of the cake across periods is described by the following \textbf{transition equation (or law of motion)}:
    \[
    W_{t+1} = W_t - c_t \tag{1}
    \]
    where \( W_t \) is the remaining cake (state), and \( c_t \) is the amount consumed in period \( t \) (control).

    \item This relation holds for each period \( t = 1, 2, 3, \dots, T \). It shows how today's choice affects tomorrow's available resources.

    \item Therefore, the decision problem is: how much of the cake should be eaten in each period to maximize total lifetime utility?

    \item Formally, how can we determine the optimal consumption path \( \{c_t\}_{t=1}^{T} = (c_1, c_2, \dots, c_T) \)? That is, the best way to spread the cake over time.

    \item There are two main approaches to solving this kind of optimization problem:
\end{itemize}

\begin{enumerate}
    \item \textbf{The Direct Method (also called the Sequence Approach)} \\
    This involves solving the problem by substituting constraints into the utility function and directly optimizing.

    \item \textbf{Dynamic Programming} \\
    This uses recursive logic, breaking the problem into smaller stages and solving it backward in time.
\end{enumerate}

\subsection*{\noindent\textbf{Direct Method / Sequence Approach}}
\addcontentsline{toc}{subsection}{Direct Method / Sequence Approach}

\begin{itemize}
    \item One way to solve this kind of constrained optimization is to directly work with all constraints at once — this is called the \textit{sequence approach}, a term popularized by Stokey and Lucas (1989).

    \item This is the method we’ve been using throughout the course so far.

    \item The setup of the optimization problem is:
    \[
    \max_{\{c_t\}_{t=1}^T, \{W_{t+1}\}_{t=2}^{T+1}} \sum_{t=1}^T \beta^{(t-1)} u(c_t), \tag{2}
    \]

    \item subject to the transition equation from before:
    \[
    W_{t+1} = W_t - c_t \quad \text{for all } t = 1, 2, \dots, T
    \]
    
    \item Additionally, we impose non-negativity constraints on both consumption and remaining cake:
    \[
    c_t \geq 0, \quad W_t \geq 0
    \]
    and the initial cake size \( W_1 \) is given.

    \item Alternatively, we can rewrite the full set of period-by-period constraints into a single \textbf{intertemporal budget constraint} by summing:
    \[
    \sum_{t=1}^T c_t + \sum_{t=1}^T W_{t+1} = \sum_{t=1}^T W_t
    \]

    \item Which simplifies to:
    \[
    \sum_{t=1}^T c_t + W_{T+1} = W_1 \tag{3}
    \]

    \item This final expression shows the total resources available (initial cake) equal the total consumed plus the cake leftover at the end.
\end{itemize}

\begin{itemize}
    \item In this setup, the non-negativity constraints are straightforward:
    \[
    c_t \geq 0 \quad \text{for } t = 1, 2, \dots, T, \quad \text{and} \quad W_{T+1} \geq 0
    \]

    \item From equation (3), we know that the sum of consumption over all periods plus the final leftover cake must equal the initial amount:
    \[
    \sum_{t=1}^T c_t + W_{T+1} = W_1
    \]

    \item Let \( \lambda \) be the multiplier on this constraint. The first-order conditions are:
    \[
    \beta^{t-1} u'(c_t) = \lambda \quad \text{for } t = 1, 2, \dots, T
    \]
    \[
    \lambda = \phi
    \]
    where \( \phi \) is the multiplier on the constraint \( W_{T+1} \geq 0 \).

    \item The non-negativity constraint \( c_t \geq 0 \) is not active here because we assume that the utility function satisfies the Inada condition — i.e., marginal utility goes to infinity as consumption approaches zero:
    \[
    \lim_{c \to 0} u'(c) = \infty
    \]

    \item So in practice, the agent always consumes something in each period.

    \item For any two periods, the FOCs imply the Euler condition. For example:
    \begin{itemize}
        \item In period 1: \( \beta^{0} u'(c_1) = u'(c_1) = \lambda \)
        \item In period 2: \( \beta^{1} u'(c_2) = \beta u'(c_2) = \lambda \)
    \end{itemize}

    \item Eliminating \( \lambda \) gives:
    \[
    u'(c_t) = \beta u'(c_{t+1})
    \]

    \item This equation characterizes the optimal intertemporal allocation — it balances marginal utility across periods, discounted by \( \beta \).

    \item Final result (Euler equation):
    \[
    \boxed{u'(c_t) = \beta u'(c_{t+1})} \tag{4}
    \]
\end{itemize}

\begin{itemize}
    \item The Euler equation is a \textbf{necessary condition} for any \( t \): if it's violated, the agent can improve utility by adjusting \( c_t \) and \( c_{t+1} \).
    
    \item To recall its meaning: suppose we have a candidate solution for this problem given by \( \{c_t^*\}_{t=1}^T, \{W_t^*\}_{t=2}^{T+1} \).
    
    \item The Euler equation tells us that the marginal utility cost of reducing consumption in period \( t \) equals the discounted marginal utility gain from consuming that saved amount in period \( t+1 \), via factor \( \beta \).

    \item \textit{But what if the deviation spans more than one period?} That is, what if the agent reduces consumption today and delays re-spending for more than one period?

    \item Example: suppose consumption is reduced by an amount \( x \) in period \( t \), saved for two periods, and then spent in \( t+2 \). Can this increase utility?

    \item This type of deviation isn't directly ruled out by the one-period Euler equation (4).

    \item However, we can chain together two Euler conditions:
    \[
    u'(c_t) = \beta u'(c_{t+1})
    \]
    \[
    u'(c_{t+1}) = \beta u'(c_{t+2})
    \]
    \item Combining them yields:
    \[
    u'(c_t) = \beta^2 u'(c_{t+2})
    \]

    \item This shows that even across two periods, marginal utility is optimally balanced, and no such multi-period deviation will raise utility — the candidate solution remains valid.
\end{itemize}

\begin{itemize}
    \item Since the problem is finite, the fact that the Euler equation holds between all adjacent periods implies that any local deviation from a candidate solution won't improve utility — as long as the Euler equation is satisfied.

    \item But this condition is only \textit{necessary}, not sufficient. A solution can satisfy the Euler equation but still be suboptimal.

    \item The true optimal solution satisfies the Euler equation in every period \textit{up to the point where the agent has fully consumed the cake}.

    \item Therefore, in an optimal plan, the non-negativity constraint on \( W_{T+1} \) must bind.

    \item Without this constraint, the agent would choose \( W_{T+1} = -\infty \), which implies they consume more than the total cake — effectively dying with debt (or owing cake!).

    \item So, for optimality, this constraint must be binding to ensure that no cake is left unused at the end.

    \item The problem is fully characterized by:
    \begin{itemize}
        \item The initial condition \( W_1 \) (the starting cake size), and
        \item The terminal condition \( W_{T+1} = 0 \) (no cake left).
    \end{itemize}

    \item Together with the \( T - 1 \) Euler equations and the intertemporal budget constraint (3), we determine the optimal consumption path.

    \item Let the value of this optimal solution be denoted:
    \[
    V_T(W_1)
    \]
    where \( T \) is the number of periods and \( W_1 \) is the initial cake size.
\end{itemize}

\begin{itemize}
    \item \( V_T(W_1) \) represents the maximum utility attainable in a \( T \)-period problem, starting with a cake of size \( W_1 \).
    
    \item From this point forward, we refer to this as the \textbf{value function}, directly analogous to the household indirect utility functions discussed earlier.

    \item The value function is defined as:
    \[
    V_T(W_1) = \max_{c_t} \sum_{t=1}^T \beta^{t-1} u(c_t) + \lambda \left[ W_1 - \sum_{t=1}^T c_t - W_{T+1} \right] + \phi W_{T+1}
    \]

    \item This formulation includes the intertemporal resource constraint and the terminal condition.

    \item As with indirect utility, a marginal increase in \( W_1 \) leads to a change in lifetime utility equal to the marginal utility of consumption — regardless of when that cake is consumed.

    \item Mathematically:
    \[
    \frac{dV(W_1)}{dW_1} = V_T'(W_1) = \lambda = \beta^{t-1} u'(c_t)
    \]

    \item This holds for any \( t = 1, 2, \dots, T \)

    \item That is, \textit{the timing of consuming the extra cake doesn’t affect utility}, as long as the consumer is optimizing — utility adjusts through the Euler condition.
\end{itemize}

\subsection*{\noindent\textbf{Summary of the direct / sequential approach}}
\addcontentsline{toc}{subsection}{Summary of the direct / sequential approach}

\begin{itemize}
    \item A unique solution exists under the following conditions:
    \begin{enumerate}
        \item The utility function is strictly increasing: \( u'(\cdot) > 0 \)
        \item The utility function is strictly concave: \( u''(\cdot) < 0 \)
        \item The constraint set is bounded (i.e., it rules out infinite values), ensuring a well-defined and finite choice set
    \end{enumerate}

    \item Additionally, the \textbf{Inada condition} ensures a unique \textit{interior} solution by avoiding corner cases:
    \[
    \lim_{c \to 0} u'(c) = \infty
    \]

    \item The resulting unique optimal consumption path \( \{c_t^*\}_{t=1}^{T} = (c_1^*, c_2^*, \dots, c_T^*) \) is characterized by:
    \begin{enumerate}
        \item The Euler equation
        \item The intertemporal budget constraint
        \item The boundary conditions:
        \[
        W_{T+1} = 0 \quad \text{and} \quad W_1 > 0
        \]
    \end{enumerate}
\end{itemize}

\subsection*{\noindent\textbf{Dynamic Programming Approach}}
\addcontentsline{toc}{subsection}{Dynamic Programming Approach}

\begin{itemize}
    \item This method uses a \textbf{recursive formulation} — solving a dynamic problem by breaking it into smaller subproblems.

    \item When the agent faces the same decision structure in every period, the \( T \)-period problem can be simplified to a 2-period recursive problem.

    \item This approach relies on the \textit{value function} derived from solving a shorter-horizon version of the problem.

    \item Given \( W_0 \), we consider the problem:
    \[
    V_T(W_0) = \max_{c_0} u(c_0) + \beta V_{T}(W_1) \tag{5}
    \]
    where
    \[
    W_1 = W_0 - c_0
    \]

    \item Here, consumption in period 0 (\( c_0 \)) determines the remaining cake \( W_1 \) that will be available from period 1 onward.

    \item Rather than selecting a full sequence of consumption decisions, the agent only needs to choose \( c_0 \), and the rest of the behavior is embedded recursively in the value function \( V_T(W_1) \).

    \item Once \( c_0 \) is chosen (and hence \( W_1 \) is known), the entire value of the problem is given by \( V_T(W_1) \).

    \item In this way, the value function completely encapsulates optimal decision-making from period 1 forward.
\end{itemize}

\begin{itemize}
    \item For the dynamic programming problem, the timing of future consumption after the initial period is irrelevant.

    \item What matters is that the agent behaves optimally starting from period 1, so utility is captured by the value function \( V_T(W_1) \).

    \item This principle is called the \textbf{principle of optimality} — the agent doesn't need to know all future values of \( c_t \) to decide on \( c_0 \); she only needs to know that future choices will be made optimally.

    \item Given this, a rational decision on \( c_0 \) can be made using just \( V_T(W_1) \), assuming optimal behavior from period 1 onward.

    \item The first-order condition, assuming differentiability of \( V_T(W_1) \), is obtained by taking the derivative with respect to the control variable \( c_0 \):
    
    \[
    V_T(W_0) = u(c_0) + \beta V_T(W_1)
            = u(c_0) + \beta V_T(W_0 - c_0)
    \]

    \[
    \Rightarrow \ u'(c_0) = \beta V_T'(W_1)
    \]

    \item Since \( W_1 = W_0 - c_0 \), the chain rule implies:
    \[
    V_T'(W_1) = V_T'(W_0 - c_0)
    \]
\end{itemize}

\begin{itemize}
    \item The marginal benefit of consuming less in period 0 is reflected in the derivative of the value function with respect to \( W_1 \).

    \item As discussed earlier in the sequence (direct) approach, we already established:
    \[
    V_T'(W_1) = \beta^{t - 1} u'(c_t) = u'(c_1) \tag{6}
    \]

    \item Therefore, we conclude:
    \[
    u'(c_0) = \beta u'(c_1)
    \]

    \item This is the familiar Euler equation.

    \item Once \( V_T'(W_1) \) is known, the entire optimal consumption path can be derived using equation (6).

    \item Of course, the simplicity here is somewhat deceptive — this step relied on knowing \( V_T(W_1) \) from solving the direct approach first.

    \item In practical application, with a finite horizon, we solve this recursively by working backward from period \( T \).
\end{itemize}

\textcolor{blue}{\textbf{\uline{Full procedure}}}

{\color{blue}
\[
V_T(W_0) = u(c_0) + \beta V_T(W_1)
\]

\[
W_1 = W_0 - c_0
\]

\[
\frac{dV_T(W_0)}{dc_0} = u'(c_0) - \beta V_T'(W_1) = 0
\]

\[
\Rightarrow u'(c_0) = \beta V_T'(W_1)
\]

\[
\text{From the direct approach:} \quad V_T'(W_1) = u'(c_1)
\]

\[
\Rightarrow u'(c_0) = \beta u'(c_1)
\]
}

\textbf{\uline{EXAMPLE}}

\begin{itemize}
    \item Consider the utility function \( u(c) = \log(c) \), and assume the agent lives only for \( T = 2 \) periods.

    \item We solve this cake-eating problem first using the \textbf{direct approach}.

    \item Lifetime utility:
    \[
    U = \log(c_1) + \beta \log(c_2)
    \]

    \item Budget constraint (with \( W_3 = 0 \)):
    \[
    c_1 + c_2 = W_1
    \]

    \item First-order conditions:
    \[
    \frac{1}{c_1} = \lambda, \quad \frac{\beta}{c_2} = \lambda, \quad c_1 + c_2 = W_1 \tag{7}
    \]

    \item Euler equation:
    \[
    \frac{1}{c_1} = \frac{\beta}{c_2} \tag{8}
    \]

    \item Solving (7) and (8):
    \[
    c_1 = \frac{W_1}{1 + \beta}, \quad c_2 = \frac{\beta W_1}{1 + \beta}
    \]

    \item Value function:
    \[
    V_T(W_1) = \log\left( \frac{W_1}{1 + \beta} \right) + \beta \log\left( \frac{\beta W_1}{1 + \beta} \right)
    \]

    \item Now solve using \textbf{dynamic programming}. At \( T = 2 \), the terminal condition is:
    \[
    W_3 = 0 = W_2 - c_2 \Rightarrow V_2(W_2) = \log(W_2)
    \]

    \item At \( T = 1 \), the value function becomes:
    \[
    V_1(W_1) = \max_{W_2} \log(W_1 - W_2) + \beta \log(W_2)
    \]

    \item First-order condition:
    \[
    \frac{1}{W_1 - W_2} = \frac{\beta}{W_2}
    \]

    \item Rearranging:
    \[
    W_2 = \frac{\beta W_1}{1 + \beta}
    \Rightarrow c_1 = W_1 - W_2 = \frac{W_1}{1 + \beta}
    \]

    \item Final value function:
    \[
    V_1(W_1) = \log\left( \frac{W_1}{1 + \beta} \right) + \beta \log\left( \frac{\beta W_1}{1 + \beta} \right)
    \]
\end{itemize}

\begin{itemize}
    \item The value function \( V_1(W_1) \) represents the \textbf{maximum utility} the agent can attain starting with cake size \( W_1 \).
    
    \item It captures the \textbf{optimal consumption allocation} between periods 1 and 2, given logarithmic preferences and discount factor \( \beta \).
    
    \item Its analytical form shows how utility depends on \textbf{initial resources} and the agent’s \textbf{degree of impatience}.
\end{itemize}


\textcolor{blue}{\textbf{\uline{Full procedure}}}

{\color{blue}
\[
\max_{c_1, c_2} \log(c_1) + \beta \log(c_2)
\quad \text{s.t.} \quad c_1 + c_2 = W_1
\]

\text{Lagrangian:}
\[
\mathcal{L} = \log(c_1) + \beta \log(c_2) - \lambda(c_1 + c_2 - W_1)
\]

\text{F.O.C.:}
\[
\frac{1}{c_1} = \lambda, \quad \frac{\beta}{c_2} = \lambda
\Rightarrow \frac{1}{c_1} = \frac{\beta}{c_2}
\]

\text{Budget:} \quad
\[
c_1 + c_2 = W_1
\]

\text{Solve:}
\[
c_1 = \frac{W_1}{1 + \beta}, \quad c_2 = \frac{\beta W_1}{1 + \beta}
\]

\text{Value function:}
\[
V_T(W_1) = \log(c_1) + \beta \log(c_2)
= \log\left( \frac{W_1}{1 + \beta} \right) + \beta \log\left( \frac{\beta W_1}{1 + \beta} \right)
\]

\text{Dynamic programming:}
\[
V_2(W_2) = \log(W_2), \quad W_3 = 0
\]

\[
V_1(W_1) = \max_{W_2} \log(W_1 - W_2) + \beta \log(W_2)
\]

\text{F.O.C.:}
\[
\frac{1}{W_1 - W_2} = \frac{\beta}{W_2}
\Rightarrow W_2 = \frac{\beta W_1}{1 + \beta}
\]

\[
c_1 = W_1 - W_2 = \frac{W_1}{1 + \beta}
\]

\text{Value function:}
\[
V_1(W_1) = \log\left( \frac{W_1}{1 + \beta} \right) + \beta \log\left( \frac{\beta W_1}{1 + \beta} \right)
\]
}

\section*{\noindent\textbf{2B.4 Infinite-Horizon Dynamic Programming}}
\addcontentsline{toc}{section}{2B.4 Infinite-Horizon Dynamic Programming}

\begin{itemize}
    \item In macroeconomics, we often analyze problems over an \textbf{infinite horizon}, where \( T = \infty \).
    
    \item For the cake-eating problem, the infinite-horizon version involves solving:
    \[
    \max_{\{c_t\}_{t=0}^\infty, \{W_{t+1}\}_{t=0}^\infty} \sum_{t=0}^{\infty} \beta^t u(c_t),
    \]
    
    \item subject to the resource constraint:
    \[
    W_{t+1} = W_t - c_t
    \]
    
    \item with the conditions:
    \[
    c_t \geq 0 \quad \text{and} \quad W_1 \text{ is given}.
    \]
    
    \item To ensure the objective function remains bounded (i.e., utility does not go to infinity), we typically require:
    \[
    0 < \beta < 1
    \]
    which imposes discounting over time and ensures convergence of the utility sum.
\end{itemize}

\begin{itemize}
    \item To formulate the dynamic programming problem, we write:
    \[
    V(W_t) = \max_{c_t, W_{t+1}} u(c_t) + \beta V(W_{t+1}) \tag{9}
    \]

    \item This equation reflects the fact that when \( T = \infty \), the agent faces the same optimization problem each period — only the cake size \( W_t \) changes over time.

    \item The function \( V(W_t) \) represents the value of the infinite-horizon problem starting with cake of size \( W_t \).

    \item In each period, the agent chooses \( c_t \), and the cake evolves according to:
    \[
    W_{t+1} = W_t - c_t
    \]

    \item The value of entering the next period with cake \( W_{t+1} \) is given by \( V(W_{t+1}) \), discounted by \( \beta \in (0, 1) \).

    \item For convenience, we could rewrite (9) more abstractly as:
    \[
    V(W) = \max_{c, W'} u(c) + \beta V(W')
    \]

    \item In this stationary infinite-horizon setup, we drop time subscripts on \( V(\cdot) \) or \( W \), since the structure is the same in every period. The prime ( \( W' \) ) simply denotes the next period's state.
\end{itemize}

\begin{itemize}
    \item In dynamic programming problems, we need to distinguish between \textbf{state variables} and \textbf{control variables}.

    \item In this problem, the state variable is the size of the cake \( W_t \), given at the start of any period.

    \item \textbf{A state variable} fully captures all past relevant information needed for the current optimization. 
    (In more complex setups, we may need to distinguish between controlled and exogenous state variables.)

    \item We don't care how we arrived at the current cake size \( W_t \); what matters is how much cake is available and how we will allocate it going forward.

    \item \textbf{A control variable} is the decision variable — here, it's the consumption \( c_t \) and consequently the leftover cake \( W_{t+1} \).

    \item The relationship between today's state and control and tomorrow's state is:
    \[
    W_{t+1} = W_t - c_t,
    \]
    which is called the \textbf{transition equation} or \textbf{law of motion}.

    \item \textit{Intuition:} We are learning to formally split what the agent controls (consumption) from what evolves (cake), allowing us to track and optimize decisions across time.
\end{itemize}

\begin{itemize}
    \item There are multiple equivalent ways to represent the cake-eating problem.

    \item[1.] Starting from equation (9), we can eliminate the transition equation and write:
    \[
    V(W_t) = \max_{c_t \in [0, W_t]} \; u(c_t) + \beta V(W_t - c_t) \tag{10}
    \]
    \item In this version, the only control variable is current consumption \( c_t \).
    \item \textit{Intuition:} This form emphasizes the direct trade-off between consuming today and leaving less for the future.

    \item[2.] Alternatively, we can reformulate the problem by making tomorrow’s state the choice variable:
    \[
    V(W_t) = \max_{W_{t+1} \in [0, W_t]} \; u(W_t - W_{t+1}) + \beta V(W_{t+1}) \tag{11}
    \]
    \item In this case, the control variable is the amount of cake left for the next period \( W_{t+1} \).
    \item \textit{Intuition:} This form highlights how the agent directly chooses how much cake to save, which often simplifies algebra and makes recursive analysis more straightforward.

    \item Either specification (10) or (11) produces the same results, but (11) is often easier to work with, so we will focus on it.
\end{itemize}

\begin{itemize}
    \item[3.] For more complicated problems, it is often easier to solve by using a \textbf{Lagrangian formulation} instead of substituting directly into the transition equation:

    \[
    V(W_t) = \max_{c_t, W_{t+1}} \Big\{ u(c_t) + \beta V(W_{t+1}) \Big\} + \lambda_t \Big[ W_t - c_t - W_{t+1} \Big],
    \]

    \item Here, both \( c_t \) and \( W_{t+1} \) are treated as control variables.

    \item The expression in (11) is called a \textbf{functional equation} or a \textbf{Bellman equation}.

    \item The unknown in the Bellman equation is the value function itself — the objective is to find a function \( V(W) \) that satisfies the condition for all \( W \).

    \item Unlike the finite-horizon setup, in the infinite horizon there is no terminal period, so we rely on the recursive structure of the value function.

    \item In general, proving the existence of such a function \( V(W) \) is nontrivial (it usually requires the advanced material of Stokey and Lucas (1989)).

    \item For this course, we do not attempt to prove existence.

    \item Instead, we assume that such a solution exists, allowing us to explore its properties.

    \item \textbf{Note:} Since \( V(W) \) appears on both sides of the Bellman equation, it acts as a fixed-point problem. This fixed-point property provides a means of solving the equation.

    \item Thanks to stationarity, we can make meaningful arguments about the existence of a value function that satisfies the functional equation.

    \item \textit{Intuition:} The Bellman equation is like a consistency check — it requires that today’s best choice, combined with tomorrow’s best choices, reproduces the same value function everywhere. It ensures the recursive logic holds.
\end{itemize}

\subsection*{\noindent\textbf{Envelope Theorem}}
\addcontentsline{toc}{subsection}{Envelope Theorem}

\begin{itemize}
    \item We can rewrite the dynamic programming problem as:
    \[
    V(W_t) = \max_{W_{t+1} \in [0, W_t]} \; u(W_t - W_{t+1}) + \beta V(W_{t+1}) \tag{12}
    \]

    \item The first-order condition for (12) is:
    \[
    u'(W_t - W_{t+1}) = \beta V'(W_{t+1})
    \]

    \item Equivalently, using \( c_t = W_t - W_{t+1} \):
    \[
    u'(c_t) = \beta V'(W_{t+1})
    \]

    \item The difficulty arises with the term \( V'(W_{t+1}) \): it involves the derivative of the value function, which is unknown.

    \item Thus, a key objective is to find a way to eliminate \( V'(W_{t+1}) \) from the first-order condition.

    \item \textit{Intuition:} The envelope theorem allows us to connect the marginal value of resources today directly to consumption, avoiding the need to explicitly differentiate the unknown value function.
\end{itemize}

\textcolor{blue}{\textbf{\uline{Full procedure of FOC (12)}}}

{\color{blue}
\[
\max_{W_{t+1}\in[0,W_t]} \; u(W_t - W_{t+1}) + \beta V(W_{t+1}) \qquad \text{(Eq. 12)}
\]

\[
\frac{d}{dW_{t+1}}\Big[ u(W_t - W_{t+1}) + \beta V(W_{t+1}) \Big] = 0
\]

\[
-\,u'(W_t - W_{t+1}) \;+\; \beta V'(W_{t+1}) \;=\; 0
\]

\[
\boxed{\,u'(W_t - W_{t+1}) \;=\; \beta\, V'(W_{t+1})\,}
\]
}

\begin{itemize}
    \item Suppose a solution exists such that the state variable evolves according to:
    \[
    W_{t+1} = g(W_t).
    \]

    \item The Bellman equation (12) can then be written as:
    \[
    V(W_t) = u(W_t - g(W_t)) + \beta V(g(W_t)).
    \]

    \item Totally differentiating with respect to \( W_t \) (where everything is treated as a function of \( W_t \)):
    \[
    V'(W_t) = u'(W_t - g(W_t)) - u'(W_t - g(W_t)) g'(W_t) + \beta V'(g(W_t)) g'(W_t)
    \]

    \item Rearranging:
    \[
    V'(W_t) = u'(W_t - W_{t+1}) - \big[ u'(W_t - W_{t+1}) - \beta V'(W_{t+1}) \big] g'(W_t)
    \]

    \item From the first-order condition, the term in brackets is zero, so we have:
    \[
    V'(W_t) = u'(W_t - W_{t+1}).
    \]

    \item Moving forward one period gives:
    \[
    V'(W_{t+1}) = u'(W_{t+1} - W_{t+2}).
    \]

    \item \textit{Intuition:} The envelope theorem allows us to replace the derivative of the unknown value function with the marginal utility of current consumption, which is directly observable.
\end{itemize}

\begin{itemize}
    \item From before, the envelope condition is:
    \[
    V'(W_{t+1}) = u'(W_{t+1} - W_{t+2}) \tag{13}
    \]

    \item Using (13), the first-order condition can be written as:
    \[
    u'(c_t) = \beta V'(W_{t+1}) = \beta u'(W_{t+1} - W_{t+2}) = \beta u'(c_{t+1})
    \]
    \[
    \Rightarrow u'(c_t) = \beta u'(c_{t+1}) \tag{14}
    \]

    \item Equation (14) is just the standard \textbf{Euler equation}.

    \item In general, the envelope condition is obtained by differentiating the problem with respect to each state variable. For this case:
    \[
    V(W_t) = \max_{W_{t+1} \in [0, W_t]} u(W_t - W_{t+1}) + \beta V(W_{t+1})
    \]

    \item Differentiating both sides of the Bellman equation with respect to \( W_t \) gives:
    \[
    V'(W_t) = u'(W_t - W_{t+1})
    \]
    (this is sometimes referred to as the Benveniste-Scheinkman condition).

    \item Shifting the envelope condition forward by one period gives equation (13), which when substituted back leads directly to the Euler equation (14).

    \item \textit{Intuition:} The envelope theorem acts as a shortcut — instead of differentiating the full optimization problem, it tells us that the marginal value of resources today equals the marginal utility of consuming them, which then delivers the Euler condition in a very clean way.
\end{itemize}

\textcolor{blue}{\textbf{\uline{Full procedure}}}

{\color{blue}
\[
\text{Bellman:}\quad V(W_t)=\max_{W_{t+1}\in[0,W_t]} \; u(W_t-W_{t+1})+\beta V(W_{t+1})
\]

\[
\text{FOC:}\quad -u'(W_t-W_{t+1})+\beta V'(W_{t+1})=0
\;\Rightarrow\;
u'(W_t-W_{t+1})=\beta V'(W_{t+1})
\]

\[
\text{Envelope:}\quad V'(W_t)=u'(W_t-W_{t+1})
\]

\[
\text{Shift:}\quad V'(W_{t+1})=u'(W_{t+1}-W_{t+2})
\]

\[
\text{Substitute:}\quad u'(W_t-W_{t+1})=\beta\,u'(W_{t+1}-W_{t+2})
\]

\[
\text{Euler:}\quad u'(c_t)=\beta\,u'(c_{t+1}),
\qquad c_t=W_t-W_{t+1},\; c_{t+1}=W_{t+1}-W_{t+2}
\]
}

\begin{itemize}
    \item The solution to the cake-eating problem satisfies the Euler equation (the necessary condition) for all \( W \).

    \item The link between consumption \( C_t \) and the next period’s cake \( W_{t+1} \) (from the different formulations) to the size of the cake (state \( W_t \)) is expressed through the policy functions:
    \[
    C_t = \mu(W_t)
    \]
    \[
    W_{t+1} = \gamma(W_t) \equiv W_t - \mu(W_t)
    \]

    \item These policy functions are \textbf{time-consistent}, meaning the optimal plan today remains optimal in the future — there is no incentive to deviate.

    \item In applied research, policy functions are crucial because they provide a direct mapping from states (resources available) to actions (consumption choices).

    \item The main objective is to solve for the value function and, from it, derive the associated policy functions.

    \item Often, closed-form solutions are not available. In such cases, we rely on \textbf{numerical methods} to approximate the value and policy functions.

    \item \textit{Intuition:} The value function tells us how good it is to have a certain amount of cake, while the policy function tells us exactly how much to eat — turning theory into actionable rules.
\end{itemize}

\section*{\noindent\textbf{2B.5 Solution Techniques}}
\addcontentsline{toc}{section}{2B.5 Solution Techniques}

\begin{itemize}
    \item There are several different methods available for solving infinite-horizon dynamic programming problems.

    \item The aim of these techniques is to compute either the \textbf{value function}, the \textbf{policy functions}, or both.

    \item The four solution techniques we will consider are:
    \begin{enumerate}
        \item Value function iteration
        \item Guess and verify method
        \item Policy function iteration
        \item Euler equation method
    \end{enumerate}

    \item The first three methods all focus on solving the functional or Bellman equation directly. 
    The Euler equation method is different: it bypasses the Bellman equation by working directly with the necessary conditions.

    \item \textit{Intuition:} Each of these techniques provides a different “toolbox” to tackle the recursive problem. Some are better for theory (guess and verify), others for computation (value or policy iteration), while the Euler approach works by exploiting first-order conditions.
\end{itemize}

\subsection*{\noindent\textbf{Value function iteration}}
\addcontentsline{toc}{subsection}{Value function iteration}

\begin{itemize}
    \item The first method constructs a sequence of value functions and associated policy functions.

    \item The idea is to iterate on the Bellman equation, starting from an arbitrary initial guess (e.g., \( V_0 = 0 \)), and continue until the sequence \( V_j \) converges.

    \item Formally, the iterative process is given by:
    \[
    V_{j+1}(W_t) = \max_{c_t, W_{t+1}} \; u(W_t, c_t) + \beta V_j(W_{t+1}),
    \]

    \item subject to the transition equation:
    \[
    W_{t+1} = g(W_t, c_t)
    \]

    \item Here, \( W_t \) is the state variable, and we can equivalently express the problem in terms of the control variable \( W_{t+1} \) or \( c_t \).

    \item \textit{Intuition:} Value function iteration works like “guessing and updating” — we start with a crude approximation of the value function and improve it step by step until it stabilizes, at which point the implied policy is optimal.
\end{itemize}

\begin{enumerate}
    \item Start with an arbitrary guess for the value function:
    \[
    V_0(W)
    \]

    \item Proceed recursively by solving:
    \[
    V_1(W_t) = \max_{W_{t+1}} \; u(W_t - W_{t+1}) + \beta V_0(W_{t+1})
    \]

    \begin{itemize}
        \item Since we now know \( V_0 \) (by construction), this maximization problem can be solved.  
        \item Find the first-order condition and solve for \( W_{t+1} \).  
        \item Call this solution \( W^* \). Substituting back gives:
        \[
        V_1(W_t) = u(W_t - W^*) + \beta V_0(W^*)
        \]
    \end{itemize}

    \item Next step:
    \[
    V_2(W_t) = \max_{W_{t+1}} \; u(W_t - W_{t+1}) + \beta V_1(W_{t+1}),
    \]
    since now \( V_1 \) is known.

    \item Continue this process to generate a sequence of value functions and policy functions.  
    Over time, this sequence converges to the true value function and its corresponding policy.
\end{enumerate}

\begin{itemize}
    \item \textit{Intuition:} This method is like “training” the value function: we start with a rough guess, refine it step by step using the Bellman equation, and eventually converge to the true optimal solution.
\end{itemize}

\subsection*{\noindent\textbf{Guess and Verify Method}}
\addcontentsline{toc}{subsection}{Guess and Verify Method}

\begin{itemize}
    \item The second method involves \textbf{guessing and verifying} a solution to the Bellman equation.

    \item The idea is to make an \textit{informed guess} about the functional form of the value function.

    \item For simple problems, the value function often shares the same general functional form as the period utility function.

    \item Example: If the period utility is logarithmic, we can often expect the value function to also take a logarithmic form.

    \item This method depends on the uniqueness of the solution to the Bellman equation.  
    However, because it relies on a good guess, it is usually only applicable in relatively simple problems.

    \item \textit{Intuition:} Guess and verify is like solving a riddle — if you guess the right “shape” of the solution, the math confirms it. But for complex problems, guessing becomes unrealistic, so this method is rarely used in practice.
\end{itemize}

\begin{enumerate}
    \item Set up the Bellman equation:
    \[
    V(W_t) = \max_{c_t, W_{t+1}} \; u(W_t, c_t) + \beta V(W_{t+1})
    \]

    \item Make a functional-form guess for the value function, e.g.:
    \[
    V(W) = XW \quad \text{where $X$ is an unknown coefficient.}
    \]

    \item Substitute the guess into the Bellman equation:
    \[
    V(W_t) = \max_{c_t, W_{t+1}} \; u(W_t, c_t) + \beta X W_{t+1}.
    \]

    \item Solve the maximization problem by finding the first-order condition and solving for \( W_{t+1} \) and \( c_t \).

    \item Substitute the solution back into the Bellman equation.  
    Check if the guess is consistent, i.e.,
    \[
    X W_t = X W_{t+1}.
    \]

    \item If a consistent \( X \) can be found, then the initial guess has been verified.  
    Otherwise, modify the guess and repeat the process.

    \item By refining successive guesses (approximations), we can converge to the correct solution for the value and policy functions, much like in value function iteration.
\end{enumerate}

\begin{itemize}
    \item \textit{Intuition:} Guess and verify works like “pattern recognition” — we assume the solution has a certain form, plug it in, and see if the math confirms it. If it fits, we’ve solved the problem; if not, we adjust and try again.
\end{itemize}

\subsection*{\noindent\textbf{Policy function iteration}}
\addcontentsline{toc}{subsection}{Policy function iteration}

\begin{itemize}
    \item This method is also known as \textbf{Howard's improvement algorithm}.

    \item The idea is to \textbf{guess a policy function} and then use it to help find the value function.

    \item Convergence is usually much faster than in the value function iteration method.

    \item For further details, see Ljungqvist and Sargent (2004), page 88.

    \item Each of the three main methods (value function iteration, guess and verify, policy function iteration) has its own advantages.

    \item However, each method is easier said than done. In practice, it is often impossible to compute even one Bellman iteration analytically.

    \item This is why computational (numerical) methods are so important in solving applied dynamic programming problems.

    \item In the next section, we will solve the cake-eating problem analytically under logarithmic preferences, using both value function iteration and guess-and-verify.

    \item Homework: Consider other economic problems that can be solved analytically.\begin{itemize}
    \item \textbf{Two-period consumption-savings model:} Solve analytically for optimal \( C_1, C_2 \) given utility \( u(C) = \log(C) \) and intertemporal budget constraint.  

    \item \textbf{Labor-leisure choice problem:} Derive optimal split between consumption and leisure under utility \( u(C,L) = \log(C) + \theta \log(L) \) with time and budget constraints.  
\end{itemize}


    \item \textit{Intuition:} Policy function iteration speeds things up by focusing directly on the “rules of behavior” (the policy functions) rather than repeatedly recalculating the value function from scratch.
\end{itemize}


\subsection*{\noindent\textbf{Euler equation approximation method}}
\addcontentsline{toc}{subsection}{Euler equation approximation method}

\begin{itemize}
    \item This method is a common alternative to the solution techniques previously discussed.

    \item The key idea: compute the policy function \textbf{directly} without solving for the value function.

    \item Advantage: we do not need the explicit form of the value function (only that it is differentiable).

    \item The approach relies on optimality conditions — such as the Euler equation and the transversality condition (TVC) — along with the resource constraint.

    \item Since these conditions are typically nonlinear, we linearize them around the steady state.  
    By solving this linear system, we obtain approximate policy functions.

    \item This produces an \textbf{approximate solution} using log-linearization.

    \item Note: this approach also applies under the direct method, which partly explains its popularity.
\end{itemize}

\bigskip

\begin{itemize}
    \item Consider the one-dimensional functional equation:
    \[
    V(x) = \max_{y \in G(x)} \{ U(x,y) + \beta V(y) \},
    \]
    where \( x \) is the state variable and \( y \) is the control variable.

    \item From earlier, the first-order condition is given by the Euler equation.

    \item However, the Euler condition alone is not sufficient.  
    We also require the \textbf{transversality condition (TVC)}:
    \[
    \lim_{t \to \infty} \beta^t U_x(x_t, x_{t+1}) x_t = 0
    \]

    \item Interpretation: the product of the marginal utility from the state \( x_t \) and the state level \( x_t \) must not grow asymptotically faster than \( 1/\beta \).

    \item Stokey–Lucas (section 4.5) show that the Euler equation + TVC are sufficient conditions for characterizing solutions to dynamic programming problems.

    \item These conditions hold under standard assumptions in neoclassical models: concave, differentiable utility and production functions; convex, nonempty choice sets.
    
    \item \textit{Intuition:} The Euler equation ensures local optimality between periods, while the TVC rules out “Ponzi schemes” (endless borrowing) by guaranteeing sustainability in the long run.
\end{itemize}

\section*{\noindent\textbf{2B.6 Solving the Cake-Eating Example}}
\addcontentsline{toc}{section}{2B.6 Solving the Cake-Eating Example}

\begin{itemize}
\item \textbf{Consider the following infinite-horizon cake-eating problem:}
\[
\max_{\{c_t\}_{t=1}^{\infty}, \{W_{t+1}\}_{t=1}^{\infty}} \sum_{t=0}^{\infty} \beta^t \log(C_t),
\]

\item \textbf{subject to} \quad $W_{t+1} = (1 + r)W_t - C_t$ \quad and a given value for $W_1$

\item \textbf{Note:} The cake that is saved (i.e., not consumed) grows at a constant rate of $1 + r$

\item \textbf{We will solve this problem using:}
\begin{enumerate}
    \item The value function iteration method
    \item The guess and verify method
    \item The Euler equation method
\end{enumerate}

\item \textbf{Intuition:} The agent allocates consumption over time to maximize utility from eating cake while accounting for the growth of uneaten cake. This trade-off gives rise to a dynamic optimization problem.
\end{itemize}

\subsection*{\noindent\textbf{The value function iteration method}}
\addcontentsline{toc}{subsection}{The value function iteration method}

\textcolor{blue}{\textbf{\uline{Full procedure}}}

{\color{blue}

\textbf{Step 0: Initial guess for value function}
\[
V_0(W) = \log(W)
\]

\textbf{Step 1: First iteration — Bellman equation}
\[
V_1(W_t) = \max_{W_{t+1}} \left\{ \log\left[(1 + r)W_t - W_{t+1}\right] + \beta \log(W_{t+1}) \right\}
\]

\textbf{FOC:}
\[
\frac{1}{(1 + r)W_t - W_{t+1}} = \frac{\beta}{W_{t+1}}
\Rightarrow W_{t+1} = \frac{\beta(1 + r)W_t}{1 + \beta}
\]

\textbf{Then:}
\[
C_t = (1 + r)W_t - W_{t+1} = \frac{(1 + r)W_t}{1 + \beta}
\]

\textbf{Substitute into } $V_1(W_t)$:
\begin{align*}
V_1(W_t) &= \log\left[(1 + r)W_t - W_{t+1}\right] + \beta \log(W_{t+1}) \\
&= \log\left[\frac{(1 + r)W_t}{1 + \beta}\right] + \beta \log\left[\frac{\beta(1 + r)W_t}{1 + \beta}\right] \\
&= (1 + \beta) \log(W_t) + (1 + \beta) \log\left(\frac{1 + r}{1 + \beta}\right) + \beta \log(\beta) \\
&= E_1 + (1 + \beta) \log(W_t)
\end{align*}

\textbf{Step 2: Second iteration}
\[
V_2(W_t) = \max_{W_{t+1}} \left\{ \log\left[(1 + r)W_t - W_{t+1}\right] + \beta V_1(W_{t+1}) \right\}
\]

\[
V_1(W_{t+1}) = E_1 + (1 + \beta) \log(W_{t+1})
\]

\textbf{Substitute:}
\[
V_2(W_t) = \max_{W_{t+1}} \left\{ \log\left[(1 + r)W_t - W_{t+1}\right] + \beta E_1 + \beta(1 + \beta) \log(W_{t+1}) \right\}
\]

\textbf{Drop constant:}
\[
\Rightarrow \max_{W_{t+1}} \left\{ \log\left[(1 + r)W_t - W_{t+1}\right] + \beta(1 + \beta) \log(W_{t+1}) \right\}
\]

\textbf{FOC:}
\[
\frac{1}{(1 + r)W_t - W_{t+1}} = \frac{\beta(1 + \beta)}{W_{t+1}}
\Rightarrow W_{t+1} = \frac{\beta(1 + \beta)(1 + r)W_t}{1 + \beta + \beta^2}
\]

\[
C_t = \frac{(1 + r)W_t}{1 + \beta + \beta^2}
\]

\textbf{Substitute into } $V_2(W_t)$:
\begin{align*}
V_2(W_t) &= \log\left[\frac{(1 + r)W_t}{1 + \beta + \beta^2}\right]
+ \beta E_1 + \beta(1 + \beta) \log\left[\frac{\beta(1 + \beta)(1 + r)W_t}{1 + \beta + \beta^2}\right] \\
&= (1 + \beta + \beta^2) \log(W_t) + \text{(constants)} \\
&= E_2 + (1 + \beta + \beta^2) \log(W_t)
\end{align*}

\textbf{General pattern after $j$ steps:}
\[
V_j(W_t) = E_j + \left( \sum_{s=0}^j \beta^s \right) \log(W_t)
\quad \text{and} \quad
W_{t+1} = \left( \frac{\sum_{s=1}^j \beta^s}{\sum_{s=0}^j \beta^s} \right)(1 + r)W_t
\]

\textbf{As } $j \to \infty$:
\[
\sum_{s=0}^\infty \beta^s = \frac{1}{1 - \beta}, \quad
\sum_{s=1}^\infty \beta^s = \frac{\beta}{1 - \beta}
\Rightarrow
W_{t+1} = \beta(1 + r)W_t, \quad
C_t = (1 - \beta)(1 + r)W_t
\]

}

\textbf{What does these policy functions tell us?}
The optimal policy is to save a constant fraction of the cake (regardless of the size of the cake) and eat the remaining fraction

\subsection*{\noindent\textbf{The guess and verify method}}
\addcontentsline{toc}{subsection}{The guess and verify method}

\textcolor{blue}{\textbf{\uline{Full procedure}}}

{\color{blue}

\textbf{Step 1: Bellman equation}
\[
V(W_t) = \max_{W_{t+1}} \left\{ \log\left[(1 + r)W_t - W_{t+1} \right] + \beta V(W_{t+1}) \right\}
\]

\textbf{Step 2: Guess a functional form}
\[
V(W) = E + F \log(W), \quad \text{where } E, F \text{ to be determined}
\]

\textbf{Step 3: Substitute guess into Bellman equation}
\[
V(W_t) = \max_{W_{t+1}} \left\{ \log\left[(1 + r)W_t - W_{t+1} \right] + \beta \left[E + F \log(W_{t+1}) \right] \right\}
\]

\textbf{Step 4: First-order condition}
\[
\frac{1}{(1 + r)W_t - W_{t+1}} = \frac{\beta F}{W_{t+1}}
\Rightarrow W_{t+1} = \frac{\beta F(1 + r)}{1 + \beta F} W_t
\]

\textbf{Step 5: Substitute $W_{t+1}$ into value function}
\begin{align*}
V(W_t) &= \log\left[ (1 + r)W_t - \frac{\beta F(1 + r)}{1 + \beta F} W_t \right] \\
&\quad + \beta E + \beta F \log\left( \frac{\beta F(1 + r)}{1 + \beta F} W_t \right) \\
&= \log\left[ \frac{(1 + r)W_t}{1 + \beta F} \right]
+ \beta E + \beta F \log\left[ \frac{\beta F(1 + r)}{1 + \beta F} W_t \right] \\
&= \log\left( \frac{1 + r}{1 + \beta F} \right)
+ \log(W_t)
+ \beta E
+ \beta F \left[ \log\left( \frac{\beta F(1 + r)}{1 + \beta F} \right) + \log(W_t) \right]
\end{align*}

\textbf{Step 6: Combine log terms}
\begin{align*}
V(W_t) &= \left[ \log\left( \frac{1 + r}{1 + \beta F} \right)
+ \beta E + \beta F \log\left( \frac{\beta F(1 + r)}{1 + \beta F} \right) \right] \\
&\quad + \left[ 1 + \beta F \right] \log(W_t)
\end{align*}

\textbf{Compare with guess:} $V(W_t) = E + F \log(W_t)$

\textbf{Match coefficients:}
\[
F = 1 + \beta F \Rightarrow F(1 - \beta) = 1 \Rightarrow F = \frac{1}{1 - \beta}
\]

\[
E = \log\left( \frac{1 + r}{1 + \beta F} \right) + \beta E + \beta F \log\left( \frac{\beta F(1 + r)}{1 + \beta F} \right)
\]

\textbf{Solve for } $E(1 - \beta)$:
\begin{align*}
E(1 - \beta) &= \log\left( \frac{1 + r}{1 + \beta F} \right)
+ \beta F \log\left( \frac{\beta F(1 + r)}{1 + \beta F} \right) \\
&= \log[(1 + r)(1 - \beta)] + \frac{\beta}{1 - \beta} \log[\beta(1 + r)] \\
&= \log(1 + r) + \log(1 - \beta)
+ \frac{\beta}{1 - \beta} \left[ \log(\beta) + \log(1 + r) \right]
\end{align*}

\textbf{Final expression for } $E$:
\[
E = \frac{1}{1 - \beta} \left[
\log(1 - \beta)
+ \frac{\beta}{1 - \beta} \log(\beta)
+ \left(1 + \frac{\beta}{1 - \beta} \right) \log(1 + r)
\right]
\]

\textbf{Value function:}
\[
V(W_t) = E + \frac{1}{1 - \beta} \log(W_t)
\]

\textbf{Policy function (from FOC):}
\[
W_{t+1} = \frac{\beta F(1 + r)}{1 + \beta F} W_t
= \beta(1 + r) W_t
\]

\textbf{Consumption:}
\[
C_t = (1 + r)W_t - W_{t+1} = (1 + r)(1 - \beta)W_t
\]

\textbf{Envelope condition (verification):}
\[
V'(W_t) = \frac{1 + r}{(1 + r)W_t - W_{t+1}}
= \frac{1 + r}{(1 + r)(1 - \beta) W_t}
= \frac{1}{(1 - \beta) W_t}
= \frac{F}{W_t}
\Rightarrow F = \frac{1}{1 - \beta}
\quad \text{(consistent)}
\]

}

\subsection*{\noindent\textbf{The Euler equation method}}
\addcontentsline{toc}{subsection}{The Euler equation method}

\textcolor{blue}{\textbf{\uline{Full procedure}}}

{\color{blue}

\textbf{Step 1: Bellman equation}
\[
V(W_t) = \max_{W_{t+1}} \left\{ \log\left[(1 + r)W_t - W_{t+1} \right] + \beta V(W_{t+1}) \right\}
\]

\textbf{Step 2: Define consumption}
\[
C_t = (1 + r)W_t - W_{t+1}
\]

\textbf{Step 3: First-order condition}
\[
\frac{1}{C_t} = \beta V'(W_{t+1})
\]

\textbf{Step 4: Envelope condition}
\[
V'(W_t) = \frac{1 + r}{C_t}
\]

\textbf{Step 5: Euler equation}
\[
\frac{1}{C_t} = \beta \cdot \frac{1 + r}{C_{t+1}} \Rightarrow C_{t+1} = \beta(1 + r) C_t
\]

\textbf{Step 6: Forward iteration}
\[
C_{t+j} = \left[ \beta(1 + r) \right]^j C_t
\]

\textbf{Step 7: Use transition law}
\[
W_t = \frac{W_{t+1}}{1 + r} + \frac{C_t}{1 + r}
\Rightarrow W_{t+1} = \frac{W_{t+2}}{1 + r} + \frac{C_{t+1}}{1 + r}
\]

\[
\Rightarrow W_t = \left( \frac{1}{1 + r} \right)^2 W_{t+2}
+ \left( \frac{1}{1 + r} \right)^2 C_{t+1}
+ \frac{C_t}{1 + r}
\]

\textbf{Step 8: Forward-substitute T times}
\[
W_t = \left( \frac{1}{1 + r} \right)^T W_{t+T}
+ \frac{1}{1 + r} \sum_{j=0}^{T-1} \left( \frac{1}{1 + r} \right)^j C_{t+j}
\]

\textbf{Step 9: Impose transversality condition (TVC)}
\[
\lim_{T \to \infty} \left( \frac{1}{1 + r} \right)^T W_{t+T} = 0
\]

\textbf{Then:}
\[
W_t = \frac{1}{1 + r} \sum_{j=0}^{\infty} \left( \frac{1}{1 + r} \right)^j C_{t+j}
= \sum_{j=0}^{\infty} \left( \frac{1}{1 + r} \right)^{j+1} C_{t+j}
\]

\textbf{Step 10: Substitute Euler path of consumption}
\[
C_{t+j} = \left[ \beta(1 + r) \right]^j C_t
\]

\[
(1 + r)W_t = \sum_{j=0}^{\infty} \left( \frac{1}{1 + r} \right)^j \left[ \beta(1 + r) \right]^j C_t
= \sum_{j=0}^{\infty} \left[ \beta \right]^j C_t
= \frac{1}{1 - \beta} C_t
\]

\textbf{Step 11: Solve for optimal consumption}
\[
C_t = (1 - \beta)(1 + r)W_t
\]

\textbf{Step 12: Use transition equation to find capital}
\[
W_{t+1} = (1 + r)W_t - C_t = (1 + r)W_t - (1 - \beta)(1 + r)W_t = \beta(1 + r)W_t
\]

}

\textbf{The Euler equation} balances the marginal benefit of consuming today with the discounted benefit of saving for future consumption.
It shows that optimal consumption grows at a \textbf{constant rate over time.}
As a result, the agent always saves a fixed fraction of wealth and consumes the rest.

\section*{\noindent\textbf{2B.7 A Little Bit of Stokey and Lucas (1989)}}
\addcontentsline{toc}{section}{2B.7 A Little Bit of Stokey and Lucas (1989)}

\begin{itemize}
\item A formal abstract treatment of dynamic programming is beyond the scope of this class.

\item Stokey and Lucas (1989) provide a rigorous mathematical treatment of deterministic and stochastic dynamic programming.

\item Here I want to simply talk about some of their key results and put them into context of what we have done (without formally covering the mathematical proof of each theorem).

\item This may help those students who wish to pursue further graduate studies where the Stokey-Lucas text is standard reading in all dynamic programming or PhD macroeconomic courses.

\item \textbf{Intuition:} This section introduces key insights from a foundational text in dynamic programming to help bridge classroom concepts with more advanced, formal graduate-level theory.
\end{itemize}

\begin{itemize}
\item As in Stokey and Lucas (1989), the Bellman equation with discounting can be expressed as:
\[
V(s) = \max_{s' \in \Gamma(s)} \sigma(s, s') + \beta V(s'). \tag{15}
\]

\item Time subscripts are removed to emphasize the stationarity of the problem. Future variables are denoted with a prime.

\item Here, $\sigma(\cdot)$ is the payoff function for the current period, $s$ is the state vector, $s'$ is the control vector, and the discount factor is $0 < \beta < 1$.

\item The policy function that defines the optimal choice of the control (i.e., the future state) given the current state is:
\[
s' = \phi(s). \tag{16}
\]

\item We are interested in solving for $\phi(s)$, which requires solving the Bellman equation (15). Thus, we must determine the value function $V(s)$ that satisfies (15).

\item It's crucial to recognize that the value function is defined as the solution to the functional equation (15).

\item Depending on the assumptions made about the payoff function and the transition function, we can determine conditions under which a solution to the Bellman equation exists.

\item \textbf{Intuition:} This slide abstracts the dynamic programming problem into its most general mathematical form. Solving the Bellman equation yields both the value function and optimal decision rules, foundational in recursive economic models.
\end{itemize}

\subsection*{\noindent\textbf{Theorem 1}}
\addcontentsline{toc}{subsection}{Theorem 1}

\begin{itemize}
\item \textbf{Theorem 1:} Assume that $\sigma(s, s')$ is real-valued, continuous, and bounded, $0 < \beta < 1$, and that the constraint set $\Gamma(s)$ is nonempty, compact-valued, and continuous. Then there exists a unique value function $V(s)$ that solves (15):
\[
V(s) = \max_{s' \in \Gamma(s)} \sigma(s, s') + \beta V(s').
\]

\item The proof of this result is given in Stokey-Lucas, Theorem 4.6 (page 79).

\item The key tools for proving Theorem 1 are:
\begin{enumerate}
    \item Contraction Mapping Theorem (SL, Theorem 3.2, page 50)
    \item Blackwell’s Sufficiency Conditions (SL, Theorem 3.3, page 54)
\end{enumerate}

\item Theorem 1 provides the theoretical foundation for the value function iteration method.

\item The idea behind Theorem 1:
\begin{itemize}
    \item Begin by guessing an initial value function $W(\cdot)$.
    \item Apply an operator $T$ to generate a new value function:
    \[
    T(W)(s) = \max_{s' \in \Gamma(s)} \sigma(s, s') + \beta W(s').
    \]
    \item This operator maps a guess $W$ into a new function $T(W)$.
\end{itemize}

\item \textbf{Intuition:} Under the right conditions, repeatedly applying the Bellman operator to any initial guess will converge to the unique value function. This is the mathematical justification behind value function iteration.

\end{itemize}

\begin{itemize}
\item Clearly, any $V(s) = T(V)(s)$ is a solution to (15). So, we can reduce the problem to finding fixed points of the operator $T(W)$.

\item If the transformation $T$ is a contraction, we can apply the contraction mapping theorem, which implies:
\begin{enumerate}
    \item There is a unique fixed point.
    \item This fixed point can be reached through iterative application of $T$, starting from any initial guess.
\end{enumerate}

\item Using the contraction mapping theorem guarantees the existence (and uniqueness) of the value function solution.

\item What ensures $T$ is a contraction? We invoke \textbf{Blackwell’s sufficiency conditions}, which require:
\begin{itemize}
    \item \textbf{Monotonicity}
    \item \textbf{Discounting}
\end{itemize}

\item \textbf{Intuition:} This slide explains how we can rigorously justify convergence of value function iteration using fixed point theory, particularly when the Bellman operator satisfies basic mathematical properties.
\end{itemize}

\subsection*{\noindent\textbf{Theorem 2}}
\addcontentsline{toc}{subsection}{Theorem 2}

\begin{itemize}
\item \textbf{Theorem 2:} If, in addition to the conditions in Theorem 1, $\sigma(s, s')$ is concave and $\Gamma(s)$ is convex, then the unique solution to (15) is strictly concave, and the associated policy function is a continuous, single-valued function.

\item The proof of this result is given in Stokey-Lucas, Theorem 4.8 (page 81).

\item We have often used the Euler equation to characterize aspects of the optimal solution.

\item To do so rigorously, we require:
\begin{enumerate}[label=(\alph*)]
    \item The strict concavity of the value function (from Theorem 2);
    \item The first-order condition of (15);
    \item That the value function is differentiable (conditions for this are given in Stokey-Lucas, Theorem 4.11, page 85).
\end{enumerate}

\item \textbf{Intuition:} This theorem ensures that not only does a solution exist, but it is well-behaved — the value function is nicely shaped, and the policy function reacts smoothly to changes in the state.
\end{itemize}



\end{document}
